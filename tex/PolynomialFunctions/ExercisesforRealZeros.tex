In Exercises \ref{prelimpolystufffirst} - \ref{prelimpolystufflast}, for the given polynomial:

\begin{itemize}
\item  Use Cauchy's Bound to find an interval containing all of the real zeros.
\item  Use the Rational Zeros Theorem to make a list of possible rational zeros.
\item  Use Descartes' Rule of Signs to list the possible number of positive and negative real zeros, counting multiplicities.
\end{itemize}


\begin{multicols}{2}
\begin{enumerate}

\item $f(x) = x^{3} - 2x^{2} - 5x + 6$ \label{prelimpolystufffirst}
\item $f(x) = x^{4} + 2x^{3} - 12x^{2} - 40x - 32$

\setcounter{HW}{\value{enumi}}
\end{enumerate}
\end{multicols}

\begin{multicols}{2}
\begin{enumerate}
\setcounter{enumi}{\value{HW}}

\item $p(z) = z^{4} - 9z^{2} - 4z + 12$
\item $p(z) = z^{3} + 4z^{2} - 11z + 6$

\setcounter{HW}{\value{enumi}}
\end{enumerate}
\end{multicols}

\begin{multicols}{2}
\begin{enumerate}
\setcounter{enumi}{\value{HW}}

\item $g(t) = t^{3} - 7t^{2} + t - 7$
\item $g(t) = -2t^{3} + 19t^{2} - 49t + 20$

\setcounter{HW}{\value{enumi}}
\end{enumerate}
\end{multicols}

\begin{multicols}{2}
\begin{enumerate}
\setcounter{enumi}{\value{HW}}

\item $f(x) = -17x^{3} + 5x^{2} + 34x - 10$
\item $f(x) = 36x^{4} - 12x^{3} - 11x^{2} + 2x + 1$

\setcounter{HW}{\value{enumi}}
\end{enumerate}
\end{multicols}

\begin{multicols}{2}
\begin{enumerate}
\setcounter{enumi}{\value{HW}}

\item $p(z) = 3z^{3} + 3z^{2} - 11z - 10$
\item $p(z) = 2z^4+z^3-7z^2-3z+3$ \label{prelimpolystufflast}


\setcounter{HW}{\value{enumi}}
\end{enumerate}
\end{multicols}


In Exercises \ref{findrealzerosexerfirst} - \ref{findrealzerosexerlast}, find the real zeros of the polynomial using the techniques specified by your instructor.  State the multiplicity of each real zero.


\begin{multicols}{2}
\begin{enumerate}
\setcounter{enumi}{\value{HW}}

\item $f(x) = x^{3} - 2x^{2} - 5x + 6$ \label{findrealzerosexerfirst}
\item $f(x) = x^{4} + 2x^{3} - 12x^{2} - 40x - 32$

\setcounter{HW}{\value{enumi}}
\end{enumerate}
\end{multicols}

\begin{multicols}{2}
\begin{enumerate}
\setcounter{enumi}{\value{HW}}

\item $p(z) = z^{4} - 9z^{2} - 4z + 12$
\item $p(z) = z^{3} + 4z^{2} - 11z + 6$

\setcounter{HW}{\value{enumi}}
\end{enumerate}
\end{multicols}

\begin{multicols}{2}
\begin{enumerate}
\setcounter{enumi}{\value{HW}}

\item $g(t) = t^{3} - 7t^{2} + t - 7$
\item $g(t) = -2t^{3} + 19t^{2} - 49t + 20$

\setcounter{HW}{\value{enumi}}
\end{enumerate}
\end{multicols}

\begin{multicols}{2}
\begin{enumerate}
\setcounter{enumi}{\value{HW}}

\item $f(x) = -17x^{3} + 5x^{2} + 34x - 10$
\item $f(x) = 36x^{4} - 12x^{3} - 11x^{2} + 2x + 1$

\setcounter{HW}{\value{enumi}}
\end{enumerate}
\end{multicols}

\begin{multicols}{2}
\begin{enumerate}
\setcounter{enumi}{\value{HW}}

\item $p(z) = 3z^{3} + 3z^{2} - 11z - 10$
\item $p(z) = 2z^4+z^3-7z^2-3z+3$

\setcounter{HW}{\value{enumi}}
\end{enumerate}
\end{multicols}

\begin{multicols}{2}
\begin{enumerate}
\setcounter{enumi}{\value{HW}}

\item $g(t) = 9t^{3} - 5t^{2} - t$
\item $g(t) = 6t^{4} - 5t^{3} - 9t^{2}$

\setcounter{HW}{\value{enumi}}
\end{enumerate}
\end{multicols}

\begin{multicols}{2}
\begin{enumerate}
\setcounter{enumi}{\value{HW}}

\item $f(x) = x^4+2x^2 - 15$
\item $f(x) = x^4-9x^2+14$

\setcounter{HW}{\value{enumi}}
\end{enumerate}
\end{multicols}

\begin{multicols}{2}
\begin{enumerate}
\setcounter{enumi}{\value{HW}}

\item $p(z) = 3z^4-14z^2-5$
\item $p(z) = 2z^4-7z^2+6$

\setcounter{HW}{\value{enumi}}
\end{enumerate}
\end{multicols}

\begin{multicols}{2}
\begin{enumerate}
\setcounter{enumi}{\value{HW}}

\item $g(t) = t^6-3t^3-10$
\item $g(t) = 2t^6-9t^3+10$

\setcounter{HW}{\value{enumi}}
\end{enumerate}
\end{multicols}

\begin{multicols}{2}
\begin{enumerate}
\setcounter{enumi}{\value{HW}}

\item $f(x) = x^5-2x^4-4x+8$
\item $f(x) = 2x^5+3x^4-18x-27$ \label{findrealzerosexerlast}

\setcounter{HW}{\value{enumi}}
\end{enumerate}
\end{multicols}

\pagebreak

In Exercises \ref{realzeroswcalcfirst} - \ref{realzeroswcalclast}, use your calculator,\footnote{You \textit{can} do these without your calculator, but it may test your mettle!} to help you find the real zeros of the polynomial.  State the multiplicity of each real zero.

\begin{enumerate}
\setcounter{enumi}{\value{HW}}

\item $f(x) = x^{5} - 60x^{3} - 80x^{2} + 960x + 2304$ \label{realzeroswcalcfirst}
\item $f(x) = 25x^{5} - 105x^{4} + 174x^{3} - 142x^{2} + 57x - 9$
\item $f(x) = 90x^{4} - 399x^{3} + 622x^{2} - 399x + 90$ \label{realzeroswcalclast}

\setcounter{HW}{\value{enumi}}
\end{enumerate}

\begin{enumerate}
\setcounter{enumi}{\value{HW}}

\item Find the real zeros of $f(x) = x^{3} - \frac{1}{12}x^{2} - \frac{7}{72}x + \frac{1}{72}$ by first finding a polynomial $q(x)$ with integer coefficients such that $q(x) = N \cdot f(x)$ for some integer $N$.  (Recall that the Rational Zeros Theorem required the polynomial in question to have integer coefficients.) Show that $f$ and $q$ have the same real zeros.

\setcounter{HW}{\value{enumi}}
\end{enumerate}

In Exercises \ref{polyequexerfirst} - \ref{polyequexerlast}, find the real solutions of the polynomial equation.  (See Example \ref{polyeqineqexample}.)

\begin{multicols}{2}
\begin{enumerate}
\setcounter{enumi}{\value{HW}}

\item  $9x^{3} = 5x^{2} + x$  \label{polyequexerfirst} 
\item $9x^{2}+5x^{3}= 6x^{4}$  

\setcounter{HW}{\value{enumi}}
\end{enumerate}
\end{multicols}

\begin{multicols}{2}
\begin{enumerate}
\setcounter{enumi}{\value{HW}}

\item $z^{3} + 6 = 2z^{2} + 5z$ 
\item $z^{4} + 2z^{3} = 12z^{2} + 40z + 32$ 

\setcounter{HW}{\value{enumi}}
\end{enumerate}
\end{multicols}


\begin{multicols}{2}
\begin{enumerate}
\setcounter{enumi}{\value{HW}}

\item $t^{3} - 7t^{2} = 7-t$ 
\item $2t^{3} = 19t^{2} - 49t + 20$ 

\setcounter{HW}{\value{enumi}}
\end{enumerate}
\end{multicols}

\begin{multicols}{2}
\begin{enumerate}
\setcounter{enumi}{\value{HW}}

\item $x^{3} + x^{2} = \dfrac{11x + 10}{3}$ 
\item $x^4+2x^2 = 15$ 


\setcounter{HW}{\value{enumi}}
\end{enumerate}
\end{multicols}

\begin{multicols}{2}
\begin{enumerate}
\setcounter{enumi}{\value{HW}}

\item $14z^{2}+5=3z^{4}$  

\item $2z^5+3z^4 = 18z + 27$ \label{polyequexerlast}  

\setcounter{HW}{\value{enumi}}
\end{enumerate}
\end{multicols}


In Exercises \ref{polyinequexerfirst} - \ref{polyinequexerlast}, solve the polynomial inequality and state your answer using interval notation.



\begin{multicols}{2}
\begin{enumerate}
\setcounter{enumi}{\value{HW}}

\item $-2x^{3} + 19x^{2} - 49x + 20 > 0$ \label{polyinequexerfirst}
\item $x^{4} - 9x^{2} \leq 4x - 12$

\setcounter{HW}{\value{enumi}}
\end{enumerate}
\end{multicols}

\begin{multicols}{2}
\begin{enumerate}
\setcounter{enumi}{\value{HW}}

\item $(z - 1)^{2} \geq 4$
\item $4z^3 \geq 3z+1$

\setcounter{HW}{\value{enumi}}
\end{enumerate}
\end{multicols}

\begin{multicols}{2}
\begin{enumerate}
\setcounter{enumi}{\value{HW}}

\item $t^4 \leq 16+4t-t^3$
\item $3t^2 + 2t < t^4$

\setcounter{HW}{\value{enumi}}
\end{enumerate}
\end{multicols}

\begin{multicols}{2}
\begin{enumerate}
\setcounter{enumi}{\value{HW}}

\item $\dfrac{x^3+2 x^2}{2} < x+2$
\item $\dfrac{x^3+20x}{8} \geq x^2 + 2$

\setcounter{HW}{\value{enumi}}
\end{enumerate}
\end{multicols}

\begin{multicols}{2}
\begin{enumerate}
\setcounter{enumi}{\value{HW}}

\item $2z^4>5z^2+3$
\item $z^6 + z^3 \geq 6$ \label{polyinequexerlast}

\setcounter{HW}{\value{enumi}}
\end{enumerate}
\end{multicols}


\newpage

In Exercises \ref{polyineqfromgraphfirst} - \ref{polyineqfromgraphlast}, use the the graph of the given polynomial function to  solve the stated inequality.

\begin{multicols}{2}
\begin{enumerate}
\setcounter{enumi}{\value{HW}}

\item  \label{polyineqfromgraphfirst} Solve $f(x) < 0$. 

\begin{mfpic}[10]{-7}{7}{-6}{6}
\axes
\tlabel[cc](7,-0.5){\scriptsize $x$}
\tlabel[cc](0.5,6){\scriptsize $y$}
\tlabel[cc](-4.5, 0.75){\scriptsize $(-6,0)$}
\tlabel[cc](5, 0.75){\scriptsize $(6,0)$}
\tlabel[cc](1, 0.75){\scriptsize $(0,0)$}
\point[4pt]{(-6,0),(0,0),(6,0)}
\xmarks{-6 step 1 until 6}
\tiny
\tlpointsep{4pt}
\axislabels {x}{{$-6 \hspace{6pt}$} -6, {$-5 \hspace{6pt}$} -5, {$-4 \hspace{6pt}$} -4, {$-3 \hspace{6pt}$} -3, {$-2 \hspace{6pt}$} -2, {$-1 \hspace{6pt}$} -1, {$1$} 1, {$2$} 2, {$3$} 3, {$4$} 4, {$5$} 5, {$6$} 6}
\normalsize
\penwd{1.25pt}
\arrow \reverse \arrow \function{-7,7,0.1}{((x**3) - 36*x)/20}
\tcaption{$y = f(x)$}
\end{mfpic}

\vfill

\columnbreak

\item Solve $g(t) > 0$.


\begin{mfpic}[20][20]{-3}{3}{-2}{5}
\axes
\tlabel[cc](3,-0.5){\scriptsize $t$}
\tlabel[cc](0.25,5){\scriptsize $y$}
\tlabel[cc](-1.75, 0.3){\scriptsize $(-2,0)$}
\tlabel[cc](0.5, 0.3){\scriptsize $(0,0)$}
\point[4pt]{(-2,0), (0,0)}
\xmarks{-2,-1, 1, 2}
\tiny
\tlpointsep{4pt}
\axislabels {x}{{$-2 \hspace{6pt}$} -2, {$-1 \hspace{6pt}$} -1, {$1$} 1, {$2$} 2}
\normalsize
\penwd{1.25pt}
\arrow \reverse \arrow \function{-3,0.3,0.1}{x*((x + 2)**3)}
\tcaption{$y = g(t)$ }
\end{mfpic}


\setcounter{HW}{\value{enumi}}
\end{enumerate}
\end{multicols}


\begin{multicols}{2}
\begin{enumerate}
\setcounter{enumi}{\value{HW}}

\item  Solve $p(z) \geq 0$ 

\begin{mfpic}[20][10]{-3}{3}{-4}{4}
\axes
\tlabel[cc](3,-0.5){\scriptsize $z$}
\tlabel[cc](0.25,4){\scriptsize $y$}
\tlabel[cc](-2, 0.75){\scriptsize $(-1,0)$}
\tlabel[cc](2, 0.75){\scriptsize $(2,0)$}
\point[4pt]{(2,0), (-1,0)}
\xmarks{-2,-1,1,2}
\tiny
\tlpointsep{4pt}
\axislabels {x}{{$-2 \hspace{6pt}$} -2, {$-1 \hspace{6pt}$} -1, {$1$} 1, {$2$} 2}
\normalsize
\penwd{1.25pt}
\arrow \reverse \arrow \function{-1.70,3.45,0.1}{(-0.4)*((x-2)**2)*(x+1)}
\tcaption{$y = p(z)$}
\end{mfpic}

\vfill

\columnbreak




\item Solve $f(x) < 0$.


\begin{mfpic}[20][10]{-2}{4}{-4}{4}
\axes
\tlabel[cc](4,-0.5){\scriptsize $x$}
\tlabel[cc](0.25,4){\scriptsize $y$}
\tlabel[cc](-0.75, 0.75){\scriptsize $\left(-\frac{1}{2},0 \right)$}
\tlabel[cc](2.5, 0.75){\scriptsize $(3,0)$}
\point[4pt]{(-0.5,0), (3,0)}
\xmarks{-1,1,2,3}
\tiny
\tlpointsep{4pt}
\axislabels {x}{ {$1$} 1, {$2$} 2, {$3$} 3}
\normalsize
\penwd{1.25pt}
\arrow \reverse \arrow \function{-1.5,3.3,0.1}{(0.5)*((x+0.5)**2)*(x-3)}
\tcaption{$y = f(x)$ }
\end{mfpic}

\setcounter{HW}{\value{enumi}}
\end{enumerate}
\end{multicols}


\begin{multicols}{2}
\begin{enumerate}
\setcounter{enumi}{\value{HW}}

\item Solve $F(s) \leq 0$.  

\begin{mfpic}[20][10]{-3}{3}{-5}{5}
\axes
\tlabel[cc](3,-0.5){\scriptsize $s$}
\tlabel[cc](0.25,5){\scriptsize $y$}
\tlabel[cc](-2, -0.75){\scriptsize $(-2,0)$}
\tlabel[cc](0.5, 0.75){\scriptsize $(0,0)$}
\point[4pt]{(-2,0), (0,0)}
\xmarks{-2,-1,1,2}
\tiny
\tlpointsep{4pt}
\axislabels {x}{ {$1$} 1, {$2$} 2}
\normalsize
\penwd{1.25pt}
\arrow \reverse \arrow \function{-3,0.65,0.1}{0-x*((x + 2)**2)}
\tcaption{$y = F(s)$}
\end{mfpic}



\vfill

\columnbreak

\item \label{polyineqfromgraphlast} Solve $G(t) \geq 0$.

\begin{mfpic}[20][10]{-3}{3}{-5}{5}
\axes
\tlabel[cc](-2, 0.75){\scriptsize $(-2,0)$}
\tlabel[cc](0.5, -0.75){\scriptsize $(0,0)$}
\tlabel[cc](3,-0.5){\scriptsize $t$}
\tlabel[cc](0.25,5){\scriptsize $y$}
\point[4pt]{(-2,0), (0,0)}
\xmarks{-2,-1,1,2}
\tiny
\tlpointsep{4pt}
\axislabels {x}{  {$2$} 2}
\normalsize
\penwd{1.25pt}
\arrow \reverse \arrow \function{-2.45,0.85,0.1}{(x**3)*((x + 2)**2)}
\tcaption{$y = G(t)$}
\end{mfpic}


\setcounter{HW}{\value{enumi}}
\end{enumerate}
\end{multicols}



\begin{enumerate}
\setcounter{enumi}{\value{HW}}

\item Use the Intermediate Value Theorem, Theorem \ref{IVT} to prove that $f(x) = x^{3} - 9x + 5$ has a real zero in each of the following intervals: $[-4, -3], [0, 1]$ and $[2, 3]$.

\item  Use the concepts of End Behavior and the Intermediate Value Theorem to prove any odd-degree polynomial function with real number coefficients has at least one real zero.

\item Find an even-degree polynomial function with real number coefficients which has no real zeros.

\item  \label{bisectionexercise} Continue  the Bisection Method as introduced on  \pageref{bisectionmethod} to approximate the real zero of $f(x) = x^5-x-1$ to three decimal places.

\item  \label{sqrt2isirrationalexercise} In this exercise, we prove $\sqrt{2}$ is an irrational number and approximate its value.  Let $f(x) = x^2-2$.

\begin{enumerate} 

\item Use Decartes' Rule of Signs to prove $f$ has exactly one positive real zero.

 \item Use the Intermediate Value Theorem to prove $f$ has a zero in $[1,2]$.

\item \label{sqrt2isirrationalexercise}  Use the Rational Zeros Theorem to prove $f$ has no rational zeros.

\item  Use the Bisection Method (see  \pageref{bisectionmethod}) to approximate the zero of $f$ on $[1,2]$ to three decimal places.

\end{enumerate}

\item  Generalize the argument given in Exercise \ref{sqrt2isirrationalexercise} to prove:

\begin{enumerate}

\item If $N$ is not the perfect square of an integer, then $\sqrt{N}$ is irrational. (HINT: Consider $f(x) = x^2-N$.)

\item  For natural numbers $n \geq 2$, if $N$ is not the perfect $n^{\text{th}}$ power of an integer, then $\sqrt[n]{N}$ is irrational. (HINT: Consider $f(x) = x^n-N$.)

\end{enumerate}

\item  In Example \ref{boxnotopex} in Section \ref{GraphsofPolynomials}, a box with no top is constructed from a $10$ inch $\times$ $12$ inch piece of cardboard by cutting out congruent squares from each corner of the cardboard and then folding the resulting tabs.  We determined the volume of that box (in cubic inches) is given by  the function$V(x) = 4x^3-44x^2+120x$, where $x$ denotes the length of the side of the square which is removed from each corner (in inches), $0 < x < 5$.  Solve the inequality $V(x) \geq 80$ analytically and interpret your answer in the context of that example.

\item  From Exercise \ref{newportaboycost} in Section \ref{GraphsofPolynomials}, $C(x) = .03x^{3} - 4.5x^{2} + 225x + 250$, for $x \geq 0$ models the cost, in dollars, to produce $x$ PortaBoy game systems. If the production budget is $\$5000$, find the number of game systems which can be produced and still remain under budget.

\item Let $f(x) = 5x^{7} - 33x^{6} + 3x^{5} - 71x^{4} - 597x^{3} + 2097x^{2} - 1971x + 567$.  With the help of your classmates, find the $x$- and $y$- intercepts of the graph of $f$.  Find the intervals on which the function is increasing, the intervals on which it is decreasing and the local extrema.  Sketch the graph of $f$, using more than one picture if necessary to show all of the important features of the graph.  

\item With the help of your classmates, create a list of five polynomials with different degrees whose real zeros cannot be found using any of the techniques in this section.

\setcounter{HW}{\value{enumi}}
\end{enumerate}
 



\newpage

\subsection{Answers}

\begin{enumerate}

\item For $f(x) = x^{3} - 2x^{2} - 5x + 6$
\begin{itemize}
\item  All of the real zeros lie in the interval $[-7,7]$
\item  Possible rational zeros are $\pm 1$, $\pm 2$, $\pm 3$, $\pm 6$
\item  There are 2 or 0 positive real zeros;  there is 1 negative real zero
\end{itemize}

\item For  $f(x) = x^{4} + 2x^{3} - 12x^{2} - 40x - 32$
\begin{itemize}
\item  All of the real zeros lie in the interval $[-41,41]$
\item  Possible rational zeros are $\pm 1$, $\pm 2$, $\pm 4$, $\pm 8$, $\pm 16$, $\pm 32$
\item  There is 1 positive real zero;  there are 3 or 1 negative real zeros
\end{itemize}

\item For  $p(z) = z^{4} - 9z^{2} - 4z + 12$
\begin{itemize}
\item  All of the real zeros lie in the interval $[-13,13]$
\item  Possible rational zeros are $\pm 1$, $\pm 2$, $\pm 3$, $\pm 4$, $\pm 6$, $\pm 12$
\item  There are 2 or 0 positive real zeros;  there are 2 or 0 negative real zeros
\end{itemize}

\item For  $p(z) = z^{3} + 4z^{2} - 11z + 6$
\begin{itemize}
\item  All of the real zeros lie in the interval $[-12,12]$
\item  Possible rational zeros are $\pm 1$, $\pm 2$, $\pm 3$, $\pm 6$
\item  There are 2 or 0 positive real zeros;  there is 1 negative real zero
\end{itemize}

\item For   $g(t) = t^{3} - 7t^{2} + t - 7$
\begin{itemize}
\item  All of the real zeros lie in the interval $[-8,8]$
\item  Possible rational zeros are $\pm 1$, $\pm 7$
\item  There are 3 or 1 positive real zeros;  there are no negative real zeros
\end{itemize}

\item For   $g(t) = -2t^{3} + 19t^{2} - 49t + 20$
\begin{itemize}
\item  All of the real zeros lie in the interval $\left[-\frac{51}{2},\frac{51}{2} \right]$
\item  Possible rational zeros are  $\pm \frac{1}{2}$, $\pm 1$, $\pm 2$, $\pm \frac{5}{2}$, $\pm 4$, $\pm 5$, $\pm 10$, $\pm 20$ 
\item  There are 3 or 1 positive real zeros;  there are no negative real zeros
\end{itemize}

\item For   $f(x) = -17x^{3} + 5x^{2} + 34x - 10$
\begin{itemize}
\item  All of the real zeros lie in the interval $[-3,3]$
\item  Possible rational zeros are $\pm \frac{1}{17}$, $\pm \frac{2}{17}$, $\pm \frac{5}{17}$, $\pm \frac{10}{17}$, $\pm 1$, $\pm 2$, $\pm 5$, $\pm 10$
\item  There are 2 or 0 positive real zeros;  there is 1 negative real zero
\end{itemize}

\item For   $f(x) = 36x^{4} - 12x^{3} - 11x^{2} + 2x + 1$
\begin{itemize}
\item  All of the real zeros lie in the interval $\left[-\frac{4}{3},\frac{4}{3}\right]$
\item  Possible rational zeros are $\pm \frac{1}{36}$, $\pm \frac{1}{18}$, $\pm \frac{1}{12}$, $\pm \frac{1}{9}$, $\pm \frac{1}{6}$, $\pm \frac{1}{4}$, $\pm \frac{1}{3}$, $\pm \frac{1}{2}$, $\pm 1$
\item  There are 2 or 0 positive real zeros;  there are 2 or 0 negative real zeros
\end{itemize}

\item For   $p(z) = 3z^{3} + 3z^{2} - 11z - 10$
\begin{itemize}
\item  All of the real zeros lie in the interval $\left[-\frac{14}{3},\frac{14}{3}\right]$
\item  Possible rational zeros are $\pm \frac{1}{3}$, $\pm \frac{2}{3}$, $\pm \frac{5}{3}$, $\pm \frac{10}{3}$, $\pm 1$, $\pm 2$, $\pm 5$, $\pm 10$
\item  There is 1 positive real zero;  there are 2 or 0 negative real zeros
\end{itemize}

\item For   $p(z) = 2z^4+z^3-7z^2-3z+3$
\begin{itemize}
\item  All of the real zeros lie in the interval $\left[-\frac{9}{2},\frac{9}{2}\right]$
\item  Possible rational zeros are  $\pm \frac{1}{2}$, $\pm 1$,  $\pm \frac{3}{2}$, $\pm 3$
\item  There are 2 or 0 positive real zeros;  there are 2 or 0 negative real zeros
\end{itemize}


\item $f(x) = x^{3} - 2x^{2} - 5x + 6$ \\ $x = -2$, $x = 1$, $x = 3$ (each has mult. 1)
\item $f(x) = x^{4} + 2x^{3} - 12x^{2} - 40x - 32$ \\ $x = -2$ (mult. 3), $x = 4$ (mult. 1)


\item $p(z) = z^{4} - 9z^{2} - 4z + 12$ \\ $z = -2$ (mult. 2), $z = 1$ (mult. 1), $z = 3$ (mult. 1)
\item $p(z) = z^{3} + 4z^{2} - 11z + 6$ \\ $z = -6$ (mult. 1), $z = 1$ (mult. 2)

\item $g(t) = t^{3} - 7t^{2} + t - 7$ \\ $t = 7$ (mult. 1)
\item $g(t) = -2t^{3} + 19t^{2} - 49t + 20$ \\ $t = \frac{1}{2}$, $t = 4$, $t = 5$ (each has mult. 1)

\item $f(x) = -17x^{3} + 5x^{2} + 34x - 10$ \\ $x = \frac{5}{17}$, $x = \pm \sqrt{2}$ (each has mult. 1)
\item $f(x) = 36x^{4} - 12x^{3} - 11x^{2} + 2x + 1$ \\ $x = \frac{1}{2}$ (mult. 2), $x = -\frac{1}{3}$ (mult. 2)

\item $p(z) = 3z^{3} + 3z^{2} - 11z - 10$ \\ $z = -2$, $z = \frac{3 \pm \sqrt{69}}{6}$ (each has mult. 1)
\item $p(z) = 2z^4+z^3-7z^2-3z+3$ \\ $z = -1$, $z = \frac{1}{2}$, $z=\pm \sqrt{3}$ (each mult. 1)

\item $g(t) = 9t^{3} - 5t^{2} - t$ \\ $t = 0$, $t = \frac{5 \pm \sqrt{61}}{18}$ (each has mult. 1)
\item $g(t) = 6t^{4} - 5t^{3} - 9t^{2}$ \\ $t = 0$ (mult. 2), $t = \frac{5 \pm \sqrt{241}}{12}$ (each has mult. 1)

\item $f(x) = x^4+2x^2 - 15$ \\ $x = \pm \sqrt{3}$ (each has mult. 1)
\item $f(x) = x^4-9x^2+14$ \\ $x = \pm \sqrt{2}$, $x = \pm \sqrt{7}$ (each has mult. 1)

\item $p(z) = 3z^4-14z^2-5$ \\ $z = \pm \sqrt{5}$ (each has mult. 1)
\item $p(z)  = 2z^4-7z^2+6$ \\  $z = \pm \frac{\sqrt{6}}{2}$, $z = \pm \sqrt{2}$ (each has mult. 1)

\item $g(t) = t^6-3t^3-10$ \\ $t = \sqrt[3]{-2} = -\sqrt[3]{2}$, $t = \sqrt[3]{5}$ (each has mult. 1)
\item $g(t) = 2t^6-9t^3+10$ \\ $t =\frac{\sqrt[3]{20}}{2} $, $t = \sqrt[3]{2}$ (each has mult. 1)


\item $f(x) = x^5-2x^4-4x+8$ \\ $x = 2$, $x = \pm \sqrt{2}$ (each has mult. 1)
\item $f(x) = 2x^5+3x^4-18x-27$ \\ $x = -\frac{3}{2}$, $x = \pm \sqrt{3}$ (each has mult. 1)

\item $f(x) = x^{5} - 60x^{3} - 80x^{2} + 960x + 2304 $ \\ $x = -4$ (mult. 3), $x = 6$ (mult. 2)


\item $f(x) = 25x^{5} - 105x^{4} + 174x^{3} - 142x^{2} + 57x - 9$ \\ $x = \frac{3}{5}$ (mult. 2), $x = 1$ (mult. 3)

\item $f(x) = 90x^{4} - 399x^{3} + 622x^{2} - 399x + 90$ \\ $x = \frac{2}{3}$, $x = \frac{3}{2}$, $x = \frac{5}{3}$, $x = \frac{3}{5}$ (each has mult. 1)


\item We choose $q(x) = 72x^{3} - 6x^{2} - 7x + 1 = 72 \cdot f(x)$.  Clearly $f(x) = 0$ if and only if $q(x) = 0$ so they have the same real zeros.  In this case, $x = -\frac{1}{3}, \; x = \frac{1}{6} \;$ and $x = \frac{1}{4}$ are the real zeros of both $f$ and $q$.


\setcounter{HW}{\value{enumi}}
\end{enumerate}


\begin{multicols}{2}
\begin{enumerate}
\setcounter{enumi}{\value{HW}}

\item  $x = 0, \frac{5\pm \sqrt{61}}{18}$
\item  $x = 0, \frac{5 \pm \sqrt{241}}{12}$

\setcounter{HW}{\value{enumi}}
\end{enumerate}
\end{multicols}

\begin{multicols}{2}
\begin{enumerate}
\setcounter{enumi}{\value{HW}}

\item $z = -2,1,3$
\item $z=-2,4$

\setcounter{HW}{\value{enumi}}
\end{enumerate}
\end{multicols}


\begin{multicols}{2}
\begin{enumerate}
\setcounter{enumi}{\value{HW}}

\item $t=7$
\item $t = \frac{1}{2}, 4, 5$

\setcounter{HW}{\value{enumi}}
\end{enumerate}
\end{multicols}

\begin{multicols}{2}
\begin{enumerate}
\setcounter{enumi}{\value{HW}}

\item $x = -2, \frac{3 \pm \sqrt{69}}{6}$

\item $x = \pm \sqrt{3}$


\setcounter{HW}{\value{enumi}}
\end{enumerate}
\end{multicols}

\begin{multicols}{2}
\begin{enumerate}
\setcounter{enumi}{\value{HW}}

\item $z = \pm \sqrt{5}$

\item $z = -\frac{3}{2}, \pm \sqrt{3}$

\setcounter{HW}{\value{enumi}}
\end{enumerate}
\end{multicols}



\begin{multicols}{2}
\begin{enumerate}
\setcounter{enumi}{\value{HW}}

\item $(-\infty, \frac{1}{2}) \cup (4, 5)$
\item $\{-2\} \cup [1, 3]$

\setcounter{HW}{\value{enumi}}
\end{enumerate}
\end{multicols}

\begin{multicols}{2}
\begin{enumerate}
\setcounter{enumi}{\value{HW}}

\item $(-\infty, -1] \cup [3, \infty)$

\item $\left\{ -\dfrac{1}{2} \right\} \cup [1, \infty)$

\setcounter{HW}{\value{enumi}}
\end{enumerate}
\end{multicols}

\begin{multicols}{2}
\begin{enumerate}
\setcounter{enumi}{\value{HW}}

\item $[-2,2]$
\item $\left(-\infty, -1 \right) \cup \left(-1, 0 \right) \cup (2, \infty)$

\setcounter{HW}{\value{enumi}}
\end{enumerate}
\end{multicols}

\begin{multicols}{2}
\begin{enumerate}
\setcounter{enumi}{\value{HW}}


\item $(-\infty, -2) \cup \left(-\sqrt{2}, \sqrt{2} \right)$
\item $\{2 \} \cup [4,\infty)$

\setcounter{HW}{\value{enumi}}
\end{enumerate}
\end{multicols}

\begin{multicols}{2}
\begin{enumerate}
\setcounter{enumi}{\value{HW}}


\item $(-\infty, -\sqrt{3}) \cup (\sqrt{3}, \infty)$
\item $(-\infty, -\sqrt[3]{3}\,) \cup (\sqrt[3]{2}, \infty)$

\setcounter{HW}{\value{enumi}}
\end{enumerate}
\end{multicols}

\begin{multicols}{2}
\begin{enumerate}
\setcounter{enumi}{\value{HW}}

\item $f(x) < 0$ on $(-\infty, -6) \cup (0, 6)$

\item $g(t) > 0$ on $(-\infty, -2) \cup (0, \infty)$

\setcounter{HW}{\value{enumi}}
\end{enumerate}
\end{multicols}

\begin{multicols}{2}
\begin{enumerate}
\setcounter{enumi}{\value{HW}}

\item $p(z) \geq 0$ on $(-\infty, -1] \cup \{ 2\}$

\item $f(x) < 0$ on $\left( -\infty, -\frac{1}{2} \right) \cup \left(-\frac{1}{2}, 3 \right)$

\setcounter{HW}{\value{enumi}}
\end{enumerate}
\end{multicols}

\begin{multicols}{2}
\begin{enumerate}
\setcounter{enumi}{\value{HW}}

\item $F(s) \leq 0$ on $\{-2\} \cup [0, \infty)$

\item $G(t) \geq 0$ on $\{-2\} \cup [0, \infty)$

\setcounter{HW}{\value{enumi}}
\end{enumerate}
\end{multicols}


\begin{enumerate}
\setcounter{enumi}{\value{HW}}

\item Since $f(-4)=-23,\; f(-3)=5,\; f(0)=5,\; f(1)=-3,\; f(2)=-5\;$ and $f(3)=5$ the Intermediate Value Theorem gives that $f(x) = x^{3} - 9x + 5$ has real zeros in the intervals $[-4, -3], [0, 1]$ and $[2, 3]$. 


\item  An odd degree polynomial  function $f$ has `mismatched' end behavior.  That is, the end behavior of $f(x)$ is either:   $x \rightarrow -\infty$, $f(x) \rightarrow -\infty$ and as $x \rightarrow \infty$, $f(x) \rightarrow \infty$  or as  $x \rightarrow -\infty$, $f(x) \rightarrow \infty$ and as $x \rightarrow \infty$, $f(x) \rightarrow -\infty$.  This means at some point, $f(x) > 0$ and at some other point $f(x) < 0$.  The Intermediate Value Theorem guarantees at least one place where $f(x) = 0$.

\item  The function $f(x) = x^2+1$ has no real zeros.

\item  $x \approx 1.167$.

\item  \begin{enumerate} 

\item  $f(x)$ has only one variation in sign, so the result follows from Descartes' Rule of Signs.

 \item $f(1) = -1<0$ and $f(2) = 2>0$ so the Intermediate Value Theorem promises a zero in $[1,2]$.

\item The Rational Zeros Theorem gives the only possible rational zeros of $f$ are $\pm 1$ and $\pm 2$.  Since $f(\pm 1) = -1$ and $f(\pm 2) = 2$, $f$ has no rational zeros.  

\item  The zero of $f$ is $\sqrt{2} \approx 1.414$. 

\end{enumerate}

\item  $V(x) \geq 80$ on $[1,5-\sqrt{5}] \cup [5+\sqrt{5}, \infty)$.  Only the portion $[1,5-\sqrt{5}]$ lies in the applied domain, however.   In the context of the problem, this says for the volume of the box to be at least 80 cubic inches, the square removed from each corner needs to have a side length of at least 1 inch, but no more than $5-\sqrt{5} \approx 2.76$ inches.

\item $C(x) \leq 5000$ on (approximately) $(-\infty, 82.18]$.  The portion of this which lies in the applied domain is $(0,82.18]$.  Since $x$ represents the number of game systems, we check $C(82) = 4983.04$ and $C(83) = 5078.11$, so to remain within the production budget, anywhere between $1$ and $82$ game systems can be produced.


\setcounter{HW}{\value{enumi}}
\end{enumerate}

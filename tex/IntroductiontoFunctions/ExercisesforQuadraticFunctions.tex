In Exercises \ref{graphquadfuncfirst} - \ref{graphquadfunclast}, graph the quadratic function.  Find the vertex and axis intercepts of each graph, if they exist.  State the domain and range, identify the maximum or minimum, and list the intervals over which the function is increasing or decreasing.  If the function is given in general form, convert it into standard form; if it is given in standard form, convert it into general form.  

\begin{multicols}{3}
\begin{enumerate}

\item $f(x) = x^{2} + 2$ \label{graphquadfuncfirst}
\item $f(x) = -(x + 2)^{2}$
\item $f(x) = x^{2} - 2x - 8$

\setcounter{HW}{\value{enumi}}
\end{enumerate}
\end{multicols}

\begin{multicols}{3}
\begin{enumerate}
\setcounter{enumi}{\value{HW}}

\item $g(t) = -2(t + 1)^{2} + 4$
\item $g(t) = 2t^2 - 4t - 1$
\item $g(t) = -3t^{2} + 4t - 7$

\setcounter{HW}{\value{enumi}}
\end{enumerate}
\end{multicols}

\begin{multicols}{3}
\begin{enumerate}
\setcounter{enumi}{\value{HW}}

\item  $h(s) = s^2 + s + 1$

\item  $h(s)  = -3s^2+5s+4$

\item $h(s) = s^{2} - \dfrac{1}{100} s - 1$ \label{graphquadfunclast}

\setcounter{HW}{\value{enumi}}
\end{enumerate}
\end{multicols}

In Exercises \ref{findformulaforquadgraphfirst} - \ref{findformulaquadgraphlast}, find a formula for each function below in the form $F(x) = a(x-h)^2+k$.

\begin{multicols}{2}

\begin{enumerate}
\setcounter{enumi}{\value{HW}}

\item $~$ \label{findformulaforquadgraphfirst}

\begin{mfpic}[15]{-5}{5}{-5}{5}
\axes
\tlabel[cc](5,-0.5){\scriptsize $x$}
\tlabel[cc](0.5,5){\scriptsize $y$}
\tlabel[cc](1, 1){\scriptsize $(0,1)$}
\tlabel[cc](-2,-3.5){\scriptsize $(-2,-3)$}
\xmarks{-4,-3,-2,-1,1,2,3,4}
\ymarks{-4,-3,-2, -1, 1,2,3,4}
\tlpointsep{4pt}
\scriptsize
\axislabels {x}{ {$-4 \hspace{7pt}$} -4, {$-3 \hspace{7pt}$} -3, {$-2 \hspace{7pt}$} -2, {$-1 \hspace{7pt}$} -1, {$1$} 1, {$2$} 2, {$3$} 3, {$4$} 4}
\axislabels {y}{{$-1$} -1,{$1$} 1, {$2$} 2, {$3$} 3, {$4$} 4, {$-2$} -2, {$-3$} -3, {$-4$} -4}
\penwd{1.25pt}
\arrow \reverse \arrow \function{-4.8,0.8,0.1}{(x+2)**2-3}
\point[4pt]{(-2,-3), (0,1)}
\tcaption{ \scriptsize$y = F(x)$}
\normalsize
\end{mfpic} 



\item $~$

\begin{mfpic}[15]{-5}{5}{-5}{5}
\axes
\tlabel[cc](5,-0.5){\scriptsize $x$}
\tlabel[cc](0.5,5){\scriptsize $y$}
\tlabel[cc](1,1){\scriptsize $(0,1)$}
\tlabel[cc](2,-1.5){\scriptsize $(2,-1)$}
\xmarks{-4,-3,-2,-1,1,2,3,4}
\ymarks{-4,-3,-2, -1, 1,2,3,4}
\tlpointsep{4pt}
\scriptsize
\axislabels {x}{ {$-4 \hspace{7pt}$} -4, {$-3 \hspace{7pt}$} -3, {$-2 \hspace{7pt}$} -2, {$-1 \hspace{7pt}$} -1, {$2$} 2,  {$4$} 4}
\axislabels {y}{{$-1$} -1,{$1$} 1, {$2$} 2, {$3$} 3, {$4$} 4, {$-2$} -2, {$-3$} -3, {$-4$} -4}
\penwd{1.25pt}
\arrow \reverse \arrow \function{-1.4,5.4,0.1}{0.5*((x-2)**2)-1}
\point[4pt]{(2,-1), (0,1)}
\tcaption{ \scriptsize$y = F(x)$}
\normalsize
\end{mfpic} 

\setcounter{HW}{\value{enumi}}

\end{enumerate}

\end{multicols}

\begin{multicols}{2}

\begin{enumerate}

\setcounter{enumi}{\value{HW}}

\item $~$

\begin{mfpic}[15]{-5}{5}{-5}{5}
\axes
\tlabel[cc](5,-0.5){\scriptsize $x$}
\tlabel[cc](0.5,5){\scriptsize $y$}
\tlabel[cc](-0.75,4.25){\scriptsize $(0,4)$}
\tlabel[cc](2.5,0.5){\scriptsize $(2,0)$}
\tlabel[cc](-3,0.5){\scriptsize $(-2,0)$}
\xmarks{-4,-3,-2,-1,1,2,3,4}
\ymarks{-4,-3,-2, -1, 1,2,3,4}
\tlpointsep{4pt}
\scriptsize
\axislabels {x}{ {$-4 \hspace{7pt}$} -4, {$-3 \hspace{7pt}$} -3, {$-1 \hspace{7pt}$} -1, {$1$} 1, {$3$} 3, {$4$} 4}
\axislabels {y}{{$-1$} -1,{$1$} 1, {$2$} 2, {$3$} 3,  {$-2$} -2, {$-3$} -3, {$-4$} -4}
\penwd{1.25pt}
\arrow \reverse \arrow \function{-3,3,0.1}{4-(x**2)}
\point[4pt]{(0,4), (-2,0), (2,0)}
\tcaption{ \scriptsize$y = F(x)$}
\normalsize
\end{mfpic} 



\item $~$  \label{findformulaquadgraphlast}

\begin{mfpic}[15]{-5}{5}{-5}{5}
\axes
\tlabel[cc](5,-0.5){\scriptsize $x$}
\tlabel[cc](0.5,5){\scriptsize $y$}
\tlabel[cc](0.8,0.5){\scriptsize $(0,0)$}
\tlabel[cc](3.75,0.5){\scriptsize $(3,0)$}
\tlabel[cc](3.75,2.5){\scriptsize $(2.5,2.5)$}
\xmarks{-4,-3,-2,-1,1,2,3,4}
\ymarks{-4,-3,-2, -1, 1,2,3,4}
\tlpointsep{4pt}
\scriptsize
\axislabels {x}{ {$-4 \hspace{7pt}$} -4, {$-3 \hspace{7pt}$} -3, {$-2 \hspace{7pt}$} -2, {$-1 \hspace{7pt}$} -1,  {$1$} 1, {$2$} 2, {$4$} 4}
\axislabels {y}{{$-1$} -1,{$1$} 1, {$2$} 2, {$3$} 3,  {$4$} 4}
\penwd{1.25pt}
\arrow \reverse \arrow \function{-0.6,3.6,0.1}{2*x*(3-x)}
\point[4pt]{(0,0), (3,0), (2.5, 2.5)}
\tcaption{ \scriptsize$y = F(x)$}
\normalsize
\end{mfpic} 


\setcounter{HW}{\value{enumi}}

\end{enumerate}

\end{multicols}

In Exercises \ref{solveineququadfirsta} - \ref{solveineququadlasta}, solve the inequality.  Write your answer using interval notation.

\begin{multicols}{2}
\begin{enumerate}
\setcounter{enumi}{\value{HW}}

\item $x^{2} + 2x - 3 \geq 0$  \label{solveineququadfirsta}
\item  $16x^2+8x+1 > 0$

\setcounter{HW}{\value{enumi}}
\end{enumerate}
\end{multicols}

\begin{multicols}{2}
\begin{enumerate}
\setcounter{enumi}{\value{HW}}


\item  $t^2+9 < 6t$
\item  $9t^2 + 16 \geq 24t$


\setcounter{HW}{\value{enumi}}
\end{enumerate}
\end{multicols}

\begin{multicols}{2}
\begin{enumerate}
\setcounter{enumi}{\value{HW}}

\item  $u^2+4 \leq 4u$
\item $u^{2} + 1 < 0$


\setcounter{HW}{\value{enumi}}
\end{enumerate}
\end{multicols}

\begin{multicols}{2}
\begin{enumerate}
\setcounter{enumi}{\value{HW}}

\item $3x^{2} \leq 11x + 4$
\item $x > x^{2}$


\setcounter{HW}{\value{enumi}}
\end{enumerate}
\end{multicols}

\begin{multicols}{2}
\begin{enumerate}
\setcounter{enumi}{\value{HW}}

\item  $2t^2-4t-1 > 0$
\item  $5t+4 \leq 3t^2$


\setcounter{HW}{\value{enumi}}
\end{enumerate}
\end{multicols}

\begin{multicols}{2}
\begin{enumerate}
\setcounter{enumi}{\value{HW}}

\item $2 \leq |x^{2} - 9| < 9$
\item $x^{2} \leq |4x - 3|$

\setcounter{HW}{\value{enumi}}
\end{enumerate}
\end{multicols}


\begin{multicols}{2}
\begin{enumerate}
\setcounter{enumi}{\value{HW}}

\item $t^{2} + t + 1 \geq 0$
\item  $t^2 \geq |t|$ 

\setcounter{HW}{\value{enumi}}
\end{enumerate}
\end{multicols}

\begin{multicols}{2}
\begin{enumerate}
\setcounter{enumi}{\value{HW}}

\item  $x |x+5| \geq -6$  
\item  $x |x-3| < 2$   \label{solveineququadlasta}

\setcounter{HW}{\value{enumi}}
\end{enumerate}
\end{multicols}




In Exercises \ref{maxprofitfirst} - \ref{maxprofitlast}, cost and price-demand functions are given.  For each scenario,


\begin{itemize}

\item  Find the profit function $P(x)$.

\item  Find the number of items which need to be sold in order to maximize profit.

\item  Find the maximum profit.

\item  Find the price to charge per item in order to maximize profit.

\item  Find and interpret break-even points.

\end{itemize}


\begin{enumerate}
\setcounter{enumi}{\value{HW}}

\item  The cost, in dollars, to produce $x$ ``I'd rather be a Sasquatch'' T-Shirts is $C(x) = 2x+26$, $x \geq 0$ and the price-demand function, in dollars per shirt,  is $p(x) = 30 - 2x$, for $0 \leq x \leq 15$. \label{maxprofitfirst}

\item  The cost, in dollars, to produce $x$ bottles of $100 \%$ All-Natural Certified Free-Trade Organic Sasquatch Tonic is $C(x) = 10x+100$, $x \geq 0$ and the price-demand function, in dollars per bottle,  is $p(x) = 35 - x$, for $0 \leq x \leq 35$.

\item  The cost, in cents, to produce $x$ cups of Mountain Thunder Lemonade at Junior's Lemonade Stand  is $C(x) = 18x + 240$, $x \geq 0$ and the price-demand function, in cents per cup,  is $p(x) = 90-3x$, for $0 \leq x \leq 30$.

\item  The daily cost, in dollars, to produce $x$ Sasquatch Berry Pies is $C(x) = 3x + 36$, $x \geq 0$ and the price-demand function, in  dollars per pie,  is $p(x) = 12-0.5x$, for $0 \leq x \leq 24$.

\item  The monthly cost, in \emph{hundreds} of dollars, to produce $x$ custom built electric scooters is $C(x) = 20x + 1000$, $x \geq 0$ and the price-demand function, in \emph{hundreds} of dollars per scooter,  is $p(x) = 140-2x$, for $0 \leq x \leq 70$. \label{maxprofitlast}

%\setcounter{HW}{\value{enumi}}
%\end{enumerate}

%\begin{enumerate}
%\setcounter{enumi}{\value{HW}}

\item The International Silver Strings Submarine Band holds a bake sale each year to fund their trip to the National Sasquatch Convention.  It has been determined that the cost in dollars of baking $x$ cookies is $C(x) = 0.1x + 25$ and that the demand function for their cookies is $p = 10 - .01x$ for $0 \leq x \leq 1000$.  How many cookies should they bake in order to maximize their profit?

\item Using data from \href{http://www.bts.gov/publications/national_transportation_statistics/html/table_04_23.html}{\underline{Bureau of Transportation Statistics}}, the average fuel economy $F(t)$ in miles per gallon for passenger cars in the US $t$ years after 1980 can be modeled by  $F(t) = -0.0076t^2+0.45t + 16$, $0 \leq t \leq 28$. Find and interpret the coordinates of the vertex of the graph of $y = F(t)$.


\item  The temperature $T$, in degrees Fahrenheit, $t$ hours after 6 AM is given by: \[ T(t) = -\frac{1}{2} t^2 + 8t+32, \quad 0 \leq t \leq 12\]

What is the warmest temperature of the day?  When does this happen?

\item  Suppose $C(x) = x^2-10x+27$ represents the costs, in \textit{hundreds}, to produce $x$ \textit{thousand} pens.  How many pens should be produced to minimize the cost?  What is this minimum cost?

\item \label{fixedperimetermaxareagarden} Skippy wishes to plant a vegetable garden along one side of his house.  In his garage, he found 32 linear feet of fencing.  Since one side of the garden will border the house, Skippy doesn't need fencing along that side.  What are the dimensions of the garden which will maximize the area of the garden?  What is the maximum  area of the garden?

\item In the situation of Example \ref{donniealpaca}, Donnie has a nightmare that one of his alpaca fell into the river.  To avoid this, he wants to move his rectangular pasture \textit{away} from the river so that all four sides of the pasture require fencing.  If the total amount of fencing available is still 200 linear feet, what dimensions maximize the area of the pasture now?  What is the maximum area?  Assuming an average alpaca requires 25 square feet of pasture, how many alpaca can he raise now?

\item What is the largest rectangular area one can enclose with 14 inches of string?


\item  The height of an object dropped from the roof of an eight story building is modeled by  by the function $h(t) = -16t^2 + 64$, $0 \leq t \leq 2$. Here,  $h(t)$ is the height of the object off the ground, in feet, $t$ seconds after the object is dropped.  How long before the object hits the ground?

\item  The height $h(t)$ in feet of a model rocket above the ground $t$ seconds after lift-off is given by the function $h(t) = -5t^2+100t$, for $0 \leq t \leq 20$.  When does the rocket reach its maximum height above the ground?  What is its maximum height?

\item  Carl's friend Jason participates in the Highland Games. In one event, the hammer throw, the height $h(t)$ in feet of the hammer above the ground $t$ seconds after Jason lets it go is modeled by the function $h(t) = -16t^2 +  22.08t + 6$.  What is the hammer's maximum height?  What is the hammer's total time in the air? Round your answers to two decimal places.

\newpage

\item Assuming no air resistance or forces other than the Earth's gravity, the height above the ground at time $t$ of a falling object is given by $s(t) = -4.9t^{2} + v_{\mbox{\scriptsize $0$}}t + s_{\mbox{\scriptsize $0$}}$ where $s$ is in meters, $t$ is in seconds, $v_{\mbox{\scriptsize $0$}}$ is the object's initial velocity in meters per second and $s_{\mbox{\scriptsize $0$}}$ is its initial position in meters.  
\label{whatgoesup}

\begin{enumerate}

\item What is the applied domain of this function?
\item Discuss with your classmates what each of $v_{\mbox{\scriptsize $0$}} > 0, \; v_{\mbox{\scriptsize $0$}} = 0$ and $v_{\mbox{\scriptsize $0$}} < 0$ would mean.
\item Come up with a scenario in which $s_{\mbox{\scriptsize $0$}} < 0$.
\item Let's say a slingshot is used to shoot a marble straight up from the ground $(s_{\mbox{\scriptsize $0$}} = 0)$ with an initial velocity of 15 meters per second.  What is the marble's maximum height above the ground?  At what time will it hit the ground?
\item If the marble is shot from the top of a 25 meter tall tower,  when does it hit the ground?
\item What would the height function be if instead of shooting the marble up off of the tower, you were to shoot it straight DOWN from the top of the tower?

\end{enumerate}


\item \label{parabolicbridgecable} The two towers of a suspension bridge are 400 feet apart.  The parabolic cable\footnote{The weight of the bridge deck forces the bridge cable into a parabola and a free hanging cable such as a power line does not form a parabola.  We shall see in Exercise \ref{catenary} in Section \ref{ExpLogApplications} what shape a free hanging cable makes.} attached to the tops of the towers is 10 feet above the point on the bridge deck that is midway between the towers.  If the towers are 100 feet tall, find the height of the cable directly above a point of the bridge deck that is 50 feet to the right of the left-hand tower.

\item On New Year's Day, Jeff started weighing himself every morning in order to have an interesting data set for this section of the book.  (Discuss with your classmates if that makes him a nerd or a geek.  Also, the professionals in the field of weight management strongly discourage weighing yourself every day.  When you focus on the number and not your overall health, you tend to lose sight of your objectives. Jeff was making a noble sacrifice for science, but you should \underline{not} try this at home.)  The whole chart would be too big to put into the book neatly, so we've decided to give only a small portion of the data to you.  This then becomes a Civics lesson in honesty, as you shall soon see.  There are two charts given below.  One has Jeff's weight for the first eight Thursdays of the year (January 1, 2009 was a Thursday and we'll count it as Day 1.) and the other has Jeff's weight for the first 10 Saturdays of the year.  

\medskip

\small

\noindent \begin{tabular}{|l|r|r|r|r|r|r|r|r|} \hline
Day \# & & & & & & & &  \\
(Thursday) & 1 & 8 & 15 & 22 & 29 & 36 & 43 & 50 \\ 
\hline 
My weight & & & & & & & & \\
in pounds & 238.2 & 237.0 & 235.6 & 234.4 & 233.0 & 233.8 & 232.8 & 232.0\\ \hline
\end{tabular}

\medskip

\noindent \begin{tabular}{|l|r|r|r|r|r|r|r|r|r|r|} \hline
Day \# & & & & & & & & & & \\
(Saturday) & 3 & 10 & 17 & 24 & 31 & 38 & 45 & 52 & 59 & 66 \\ 
\hline 
My weight & & & & & & & & & & \\
in pounds & 238.4 & 235.8 & 235.0 & 234.2 & 236.2 & 236.2 & 235.2 & 233.2 & 236.8 & 238.2\\ \hline
\end{tabular}

\normalsize

\medskip

\begin{enumerate}

\item Find the least squares line for the Thursday data and comment on its goodness of fit.
\item Find the least squares line for the Saturday data and comment on its goodness of fit.
\item Use Quadratic Regression to find a parabola which models the Saturday data and comment on its goodness of fit.
\item Compare and contrast the predictions the three models make for Jeff's weight on January 1, 2010 (Day \#366).  Can any of these models be used to make a prediction of Jeff's weight 20 years from now?  Explain your answer.
\item Why is this a Civics lesson in honesty?  Well, compare the two linear models you obtained above.  One was a good fit and the other was not, yet both came from careful selections of real data.  In presenting the tables to you, we've  not lied about Jeff's weight, nor have you used any bad math to falsify the predictions.  The word we're looking for here is `disingenuous'.  Look it up and then discuss the implications this type of data manipulation could have in a larger, more complex, politically motivated setting.  

\end{enumerate}

\item (Data that is neither linear nor quadratic.)  We'll close this exercise set with two data sets that, for reasons presented later in the book, cannot be modeled correctly by lines or parabolas.  It is a good exercise, though, to see what happens when you attempt to use a linear or quadratic model when it's not appropriate.

\begin{enumerate}

\item \label{APLcats} This first data set came from a Summer 2003 publication of the Portage County Animal Protective League called ``Tattle Tails''.  They make the following statement and then have a chart of data that supports it. ``It doesn't take long for two cats to turn into 80 million.  If two cats and their surviving offspring reproduced for ten years, you'd end up with 80,399,780 cats.''  We assume $N(0) = 2$.

\medskip

\scriptsize

\noindent \begin{tabular}{|l|r|r|r|r|r|r|r|r|r|r|} \hline
Year $x$ & 1 & 2 & 3 & 4 & 5 & 6 & 7 & 8 & 9 & 10 \\ 
\hline 
Number of  & & & & & & & & & & \\
Cats $N(x)$ & 12 & 66 & 382 & 2201 & 12680 & 73041 & 420715 & 2423316 & 13968290 & 80399780 \\ \hline
\end{tabular}

\normalsize

\medskip

\noindent Use Quadratic Regression to find a parabola which models this data and comment on its goodness of fit. (Spoiler Alert: Does anyone know what type of function we need here?)

\medskip

\item \label{regsunlight} This next data set comes from the \href{http://aa.usno.navy.mil/data/docs/RS_OneYear.php}{\underline{U.S. Naval Observatory}}.  That site has loads of awesome stuff on it, but for this exercise I used the sunrise/sunset times in Fairbanks, Alaska for 2009 to give you a chart of the number of hours of daylight they get on the $21^{\mbox{st}}$ of each month.  We'll let $x = 1$ represent January 21, 2009, $x = 2$ represent February 21, 2009, and so on.

\medskip

\small

\noindent \begin{tabular}{|l|r|r|r|r|r|r|r|r|r|r|r|r|} \hline
Month  & & & & & & & & & & & & \\
Number & 1 & 2 & 3 & 4 & 5 & 6 & 7 & 8 & 9 & 10 & 11 & 12\\ 
\hline 
Hours of  & & & & & & & & & & & & \\
Daylight & 5.8 & 9.3 & 12.4 & 15.9 & 19.4 & 21.8 & 19.4 & 15.6 & 12.4 & 9.1 & 5.6 & 3.3 \\ \hline
\end{tabular}

\normalsize

\medskip

\noindent Use Quadratic Regression to find a parabola which models this data and comment on its goodness of fit. (Spoiler Alert: Does anyone know what type of function we need here?)

\end{enumerate}


\item Redraw the three scenarios discussed in the discriminant box for $a<0$.  \label{redrawthezeroscenarios}

\item Graph $f(x) = |1 - x^{2}|$

\item  Find all of the points on the line $y=1-x$ which are $2$ units from $(1,-1)$.

\item  Let $L$ be the line $y = 2x+1$.  Find a function $D(x)$ which measures the distance \textit{squared} from a point on $L$ to $(0,0)$.  Use this to find the point on $L$ closest to $(0,0)$.

\item With the help of your classmates, show that if a quadratic function $f(x) = ax^{2} + bx + c$ has two real zeros then the $x$-coordinate of the vertex is the midpoint of the zeros.

\item  \label{avoidcompsquare}  On page \pageref{standardtogeneraldiscussion}, we argued that any quadratic function in standard form $f(x) = a(x-h)^2+k$ can be converted to a quadratic function in general form $f(x) = ax^2+bx+c$ by making the identifications $b=-2ah$ and $c = ah^2+k$.  In this exercise, we use same identifications to show every parabola given in general form can be converted to standard form without completing the square.

\smallskip

Solve $b=-2ah$ for $h$ and substitute the result into the equation $c = ah^2+k$ and then solve for $k$.  Show  $h = -\frac{b}{2a}$ and $k = \frac{4ac-b^2}{4a}$ so that \[ f(x) = ax^2+bx+c = a\left(x + \dfrac{b}{2a}\right)^2  + \dfrac{4ac - b^2}{4a}. \]

\setcounter{HW}{\value{enumi}}
\end{enumerate}

In Exercises \ref{solvequadvarifirst} - \ref{solvequadvarilast}, solve the quadratic equation for the indicated variable.

\begin{multicols}{2}
\begin{enumerate}
\setcounter{enumi}{\value{HW}}


\item $x^{2} - 10y^{2} = 0$ for $x$ \label{solvequadvarifirst}
\item $y^{2} - 4y = x^{2} - 4$ for $x$

\setcounter{HW}{\value{enumi}}
\end{enumerate}
\end{multicols}

\begin{multicols}{2}
\begin{enumerate}
\setcounter{enumi}{\value{HW}}

\item $x^{2} - mx = 1$ for $x$
\item $y^{2} - 3y = 4x$ for $y$

\setcounter{HW}{\value{enumi}}
\end{enumerate}
\end{multicols}

\begin{multicols}{2}
\begin{enumerate}
\setcounter{enumi}{\value{HW}}



\item $y^{2} - 4y = x^{2} - 4$ for $y$
\item $-gt^{2} + v_{\mbox{\scriptsize $0$}}t + s_{\mbox{\scriptsize $0$}} = 0$ for $t$ (Assume $g \neq 0$.) \label{solvequadvarilast}

\setcounter{HW}{\value{enumi}}
\end{enumerate}
\end{multicols}

\begin{enumerate}
\setcounter{enumi}{\value{HW}}



\item \label{LagrangeQuadExercise} (This is a follow-up to Exercise \ref{LagrangeLinearExercise} in Section \ref{ConstantandLinearFunctions}.) The \href{https://en.wikipedia.org/wiki/Lagrange_polynomial}{\underline{Lagrange Interpolate}} function $L$ for three points $(x_{0}, y_{0})$, $(x_{1}, y_{1})$, and  $(x_{2}, y_{2})$ where $x_{0}$,  $x_{1}$, and $x_{2}$ are three distinct real numbers  is given by: \[L(x) = y_{0}  \dfrac{(x - x_{1}) (x - x_{2}) }{(x_{0} - x_{1})(x_{0} - x_{2})}+ y_{1}  \dfrac{(x - x_{0}) (x - x_{2}) }{(x_{1} - x_{0})(x_{1} - x_{2})} +  y_{2}  \dfrac{(x - x_{0}) (x - x_{1}) }{(x_{2} - x_{0})(x_{2} - x_{1})}\]

\begin{enumerate}

\item For each of the following sets of points,  find  $L(x)$ using the formula above and verify each of the points lies on the graph of $y = L(x)$.

\begin{multicols}{3}

\begin{enumerate}

\item  $(-1,1)$, $(1,1)$, $(2,4)$ % $L(x) = x^2$

\item  $(1,3)$, $(2,10)$, $(3,21)$ % $L(x) = 2x^2+x$

\item  $(0,1)$,  $(1,5)$, $(2,7)$ % $L(x) = -x^2+5x+1$



\end{enumerate}

\end{multicols}

\item  Verify that, in general, $L(x_{0}) = y_{0}$,  $L(x_{1}) = y_{1}$, and $L(x_{2}) = y_{2}$.

\item Find $L(x)$ for the points $(-1, 6)$, $(1, 4)$ and $(3,2)$.  What happens?  %$L(x) = -x+5$

\item  Under what conditions will $L(x)$ produce a quadratic function?  Make a conjecture, test some cases, and prove your answer.


\end{enumerate}

\setcounter{HW}{\value{enumi}}
\end{enumerate}

\newpage

\subsection{Answers}

\begin{enumerate}

\item \begin{multicols}{2} \raggedcolumns
$f(x) = x^{2} + 2$ (this is both forms!) \\
No $x$-intercepts \\
$y$-intercept $(0, 2)$\\
Domain: $(-\infty, \infty)$ \\
Range: $[2, \infty)$ \\
Decreasing on $(-\infty, 0]$ \\
Increasing on $[0, \infty)$ \\
Vertex $(0, 2)$ is a minimum \\
Axis of symmetry $x = 0$ \\

\begin{mfpic}[15][10]{-3}{3}{-1}{11}
\axes
\tlabel[cc](3,-0.5){\scriptsize $x$}
\tlabel[cc](0.5,11){\scriptsize $y$}
\xmarks{-2,-1,1,2}
\ymarks{1 step 1 until 10}
\tlpointsep{4pt}
\scriptsize
\axislabels {x}{{$-2 \hspace{6pt}$} -2, {$-1 \hspace{6pt}$} -1, {$1$} 1, {$2$} 2}
\axislabels {y}{{$1$} 1, {$2$} 2, {$3$} 3, {$4$} 4, {$5$} 5, {$6$} 6, {$7$} 7, {$8$} 8, {$9$} 9, {$10$} 10}
\normalsize
\point[4pt]{(0,2)}
\penwd{1.25pt}
\arrow \reverse \arrow \function{-3,3,0.1}{x**2 + 2}
\end{mfpic}

\end{multicols}

\item \begin{multicols}{2} \raggedcolumns
$f(x) = -(x + 2)^{2} = -x^2-4x-4$\\
$x$-intercept $(-2, 0)$ \\
$y$-intercept $(0, -4)$\\
Domain: $(-\infty, \infty)$ \\
Range: $(-\infty, 0]$ \\
Increasing on $(-\infty, -2]$ \\
Decreasing on $[-2, \infty)$ \\
Vertex $(-2, 0)$ is a maximum \\
Axis of symmetry $x = -2$ \\

\begin{mfpic}[15][10]{-5}{1}{-9}{1}
\axes
\tlabel[cc](1,-0.5){\scriptsize $x$}
\tlabel[cc](0.5,1){\scriptsize $y$}
\xmarks{-4,-3,-2,-1}
\ymarks{-8 step 1 until -1}
\tlpointsep{4pt}
\scriptsize
\axislabels {x}{{$-4 \hspace{6pt}$} -4, {$-3 \hspace{6pt}$} -3, {$-2 \hspace{6pt}$} -2, {$-1 \hspace{6pt}$} -1}
\axislabels {y}{{$-8$} -8, {$-7$} -7, {$-6$} -6, {$-5$} -5, {$-4$} -4, {$-3$} -3, {$-2$} -2, {$-1$} -1}
\normalsize
\point[4pt]{(-2,0), (0,-4)}
\penwd{1.25pt}
\arrow \reverse \arrow \function{-5,1,0.1}{-((x + 2)**2)}
\end{mfpic}

\end{multicols}

\item \begin{multicols}{2} \raggedcolumns
$f(x) = x^{2} - 2x - 8 = (x - 1)^{2} - 9$\\
$x$-intercepts $(-2, 0)$ and $(4, 0)$\\
$y$-intercept $(0, -8)$\\
Domain: $(-\infty, \infty)$ \\
Range: $[-9, \infty)$ \\
Decreasing on $(-\infty, 1]$ \\
Increasing on $[1, \infty)$ \\
Vertex $(1, -9)$ is a minimum \\
Axis of symmetry $x = 1$ \\

\begin{mfpic}[15][10]{-3}{5}{-10}{3}
\axes
\tlabel[cc](5,-0.5){\scriptsize $x$}
\tlabel[cc](0.5,3){\scriptsize $y$}
\xmarks{-2 step 1 until 4}
\ymarks{-9 step 1 until 2}
\tlpointsep{4pt}
\scriptsize
\axislabels {x}{{$-2 \hspace{6pt}$} -2, {$-1 \hspace{6pt}$} -1, {$1$} 1, {$2$} 2, {$3$} 3, {$4$} 4}
\axislabels {y}{{$-9$} -9, {$-8$} -8, {$-7$} -7, {$-6$} -6, {$-5$} -5, {$-4$} -4, {$-3$} -3, {$-2$} -2, {$-1$} -1, {$1$} 1, {$2$} 2}
\normalsize
\point[4pt]{(-2,0),(0,-8),(1,-9),(4,0)}
\penwd{1.25pt}
\arrow \reverse \arrow \function{-2.4,4.4,0.1}{x**2 - 2*x - 8}
\end{mfpic}

\end{multicols}

\item \begin{multicols}{2} \raggedcolumns
$g(t) = -2(t + 1)^{2} + 4 = -2t^2-4t+2$\\
$t$-intercepts {\small $(-1 - \sqrt{2}, 0)$ and $(-1 + \sqrt{2}, 0)$}\\
$y$-intercept $(0, 2)$\\
Domain: $(-\infty, \infty)$ \\
Range: $(-\infty, 4]$ \\
Increasing on $(-\infty, -1]$ \\
Decreasing on $[-1, \infty)$ \\
Vertex $(-1, 4)$ is a maximum \\
Axis of symmetry $t = -1$ \\

\begin{mfpic}[20][10]{-3.5}{2}{-5}{5}
\axes
\tlabel[cc](2,-0.5){\scriptsize $t$}
\tlabel[cc](0.5,5){\scriptsize $y$}
\xmarks{-3 step 1 until 1}
\ymarks{-4 step 1 until 4}
\tlpointsep{4pt}
\scriptsize
\axislabels {x}{{$-3 \hspace{6pt}$} -3, {$-2 \hspace{6pt}$} -2, {$-1 \hspace{6pt}$} -1, {$1$} 1}
\axislabels {y}{{$-4$} -4, {$-3$} -3, {$-2$} -2, {$-1$} -1, {$1$} 1, {$2$} 2, {$3$} 3, {$4$} 4}
\normalsize
\point[4pt]{(-2.4142,0),(0,2),(-1,4),(.4142,0)}
\penwd{1.25pt}
\arrow \reverse \arrow \function{-3.1,1.1,0.1}{4- 2*((x + 1)**2)}
\end{mfpic}

\end{multicols}



\item \begin{multicols}{2} \raggedcolumns
$g(t) = 2t^2-tx-1 = 2(t-1)^2-3$\\
$t$-intercepts {\small $\left(\frac{2-\sqrt{6}}{2}, 0\right)$ and $\left(\frac{2+\sqrt{6}}{2}, 0\right)$}\\
$y$-intercept $(0, -1)$\\
Domain: $(-\infty, \infty)$ \\
Range: $[-3, \infty)$ \\
Increasing on $[1,\infty)$ \\
Decreasing on $(-\infty,1]$ \\
Vertex $(1, -3)$ is a minimum \\
Axis of symmetry $t = 1$ \\

\begin{mfpic}[15]{-2}{4}{-4}{5}
\axes
\tlabel[cc](4,-0.5){\scriptsize $t$}
\tlabel[cc](0.5,5){\scriptsize $y$}
\xmarks{-1 step 1 until 3}
\ymarks{-3 step 1 until 4}
\tlpointsep{4pt}
\scriptsize
\axislabels {x}{{$-1 \hspace{6pt}$} -1, {$1$} 1, {$2$} 2, {$3$} 3}
\axislabels {y}{{$-3$} -3, {$-2$} -2, {$-1$} -1, {$1$} 1, {$2$} 2, {$3$} 3, {$4$} 4}
\normalsize
\point[4pt]{(-0.2247,0),(0,-1),(1,-3),(2.2247,0)}
\penwd{1.25pt}
\arrow \reverse \arrow \function{-0.8,2.8,0.1}{2*(x**2)-4*x-1}
\end{mfpic}

\end{multicols}


\item \begin{multicols}{2} \raggedcolumns 
$g(t) = -3t^{2} + 4t - 7 = -3\left(t - \frac{2}{3} \right)^{2} - \frac{17}{3}$\\
No $t$-intercepts \\
$y$-intercept $(0, -7)$\\
Domain: $(-\infty, \infty)$ \\
Range: $\left(-\infty, -\frac{17}{3}\right]$ \\
Increasing on $\left(-\infty, \frac{2}{3}\right]$ \\
Decreasing on $\left[\frac{2}{3}, \infty\right)$ \\
Vertex $\left(\frac{2}{3}, -\frac{17}{3}\right)$ is a maximum \\
Axis of symmetry $t = \frac{2}{3}$ \\

\begin{mfpic}[20][10]{-1}{3}{-15}{1}
\axes
\tlabel[cc](3,-0.5){\scriptsize $t$}
\tlabel[cc](0.25,1){\scriptsize $y$}
\xmarks{1,2}
\ymarks{-14 step 1 until -1}
\tlpointsep{4pt}
\scriptsize
\axislabels {x}{{$1$} 1, {$2$} 2}
\axislabels {y}{{$-14$} -14, {$-13$} -13, {$-12$} -12, {$-11$} -11, {$-10$} -10, {$-9$} -9, {$-8$} -8, {$-7$} -7, {$-6$} -6, {$-5$} -5, {$-4$} -4, {$-3$} -3, {$-2$} -2, {$-1$} -1}
\normalsize
\point[4pt]{(0,-7),(.66667,-5.66667)}
\penwd{1.25pt}
\arrow \reverse \arrow \function{-1,2.33,0.1}{-3*(x**2) + 4*x - 7}
\end{mfpic}

\end{multicols}

\item \begin{multicols}{2} \raggedcolumns 
$h(s) = s^2+s+1 = \left(s + \frac{1}{2}\right)^{2} + \frac{3}{4}$\\
No $s$-intercepts \\
$y$-intercept $(0, 1)$\\
Domain: $(-\infty, \infty)$ \\
Range: $\left[ \frac{3}{4}, \infty\right)$ \\
Increasing on $\left[-\frac{1}{2}, \infty\right)$ \\
Decreasing on $\left(-\infty, -\frac{1}{2}\right]$ \\
Vertex $\left(-\frac{1}{2}, \frac{3}{4}\right)$ is a minimum \\
Axis of symmetry $s = -\frac{1}{2}$ \\

\begin{mfpic}[18]{-3}{2}{-1}{5}
\axes
\tlabel[cc](2,-0.5){\scriptsize $s$}
\tlabel[cc](0.5,5){\scriptsize $y$}
\xmarks{-2,-1,1}
\ymarks{1,2,3,4}
\tlpointsep{4pt}
\scriptsize
\axislabels {x}{{$-2 \hspace{6pt}$} -2,{$-1 \hspace{6pt}$} -1,{$1$} 1}
\axislabels {y}{{$1$} 1, {$2$} 2, {$3$} 3, {$4$} 4}
\normalsize
\point[4pt]{(0,1),(-0.5,0.75)}
\penwd{1.25pt}
\arrow \reverse \arrow \function{-2.5,1.5,0.1}{(x**2)+x+1}
\end{mfpic}

\end{multicols}

\pagebreak

\item \begin{multicols}{2} \raggedcolumns
$h(s) = -3s^2+5s+4 = -3\left(s-\frac{5}{6}\right)^2 + \frac{73}{12}$\\
$s$-intercepts {\small $\left(\frac{5 - \sqrt{73}}{6}, 0\right)$ and $\left(\frac{5+\sqrt{73}}{6}, 0\right)$}\\
$y$-intercept $(0, 4)$\\
Domain: $(-\infty, \infty)$ \\
Range: $\left(-\infty,  \frac{73}{12} \right]$ \\
Increasing on $\left(-\infty, \frac{5}{6}\right]$ \\
Decreasing on $\left[ \frac{5}{6}, \infty\right)$ \\
Vertex $\left(\frac{5}{6}, \frac{73}{12} \right)$ is a maximum \\
Axis of symmetry $s = \frac{5}{6}$ \\

\begin{mfpic}[15]{-2}{4}{-4}{7}
\axes
\tlabel[cc](4,-0.5){\scriptsize $s$}
\tlabel[cc](0.5,7){\scriptsize $y$}
\xmarks{-1 step 1 until 3}
\ymarks{-3 step 1 until 6}
\tlpointsep{4pt}
\scriptsize
\axislabels {x}{{$-1 \hspace{6pt}$} -1, {$1$} 1, {$2$} 2, {$3$} 3}
\axislabels {y}{{$-3$} -3, {$-2$} -2, {$-1$} -1, {$1$} 1, {$2$} 2, {$3$} 3, {$4$} 4, {$5$} 5, {$6$} 6}
\normalsize
\point[4pt]{(-0.5907,0),(0,4),(0.8333,6.0833),(2.2573,0)}
\penwd{1.25pt}
\arrow \reverse \arrow \function{-1,2.62,0.1}{0-3*(x**2)+5*x+4}
\end{mfpic}

\end{multicols}

\item \begin{multicols}{2} \raggedcolumns
$h(s) = s^{2} - \frac{1}{100} s - 1 = \left(s - \frac{1}{200}\right)^{2} - \frac{40001}{40000}$\\
$s$-intercepts $\left(\frac{1 + \sqrt{40001}}{200}\right)$ and $\left(\frac{1 - \sqrt{40001}}{200}\right)$\\
$y$-intercept $(0, -1)$\\
Domain: $(-\infty, \infty)$ \\
Range: $\left[-\frac{40001}{40000}, \infty \right)$ \\
Decreasing on $\left(-\infty, \frac{1}{200}\right]$ \\
Increasing on $\left[\frac{1}{200}, \infty \right)$ \\
Vertex $\left(\frac{1}{200}, -\frac{40001}{40000}\right)$ is a minimum\footnote{You'll need to use your calculator to zoom in far enough to see that the vertex is not the $y$-intercept.} \\
Axis of symmetry $s = \frac{1}{200}$ \\

\begin{mfpic}[15][10]{-3}{3}{-2}{9}
\axes
\tlabel[cc](3,-0.5){\scriptsize $s$}
\tlabel[cc](0.5,9){\scriptsize $y$}
\xmarks{-2,-1,1,2}
\ymarks{1 step 1 until 8}
\tlpointsep{4pt}
\scriptsize
\axislabels {x}{{$-2 \hspace{6pt}$} -2, {$-1 \hspace{6pt}$} -1, {$1$} 1, {$2$} 2}
\axislabels {y}{{$1$} 1, {$2$} 2, {$3$} 3, {$4$} 4, {$5$} 5, {$6$} 6, {$7$} 7, {$8$} 8}
\normalsize
\point[4pt]{(0,-1), (0.005, -1.000025)}
\penwd{1.25pt}
\arrow \reverse \arrow \function{-3,3,0.1}{x**2 - (x/100) - 1}
\end{mfpic}

\end{multicols}
\setcounter{HW}{\value{enumi}}
\end{enumerate}

\begin{multicols}{2}
\begin{enumerate}
\setcounter{enumi}{\value{HW}}

\item $F(x) = (x+2)^2-3$  \vphantom{$F(x) = \frac{1}{2}(x-2)^2-1$}

\item $F(x) = \frac{1}{2}(x-2)^2-1$

\setcounter{HW}{\value{enumi}}
\end{enumerate}
\end{multicols}

\begin{multicols}{2}
\begin{enumerate}
\setcounter{enumi}{\value{HW}}

\item $F(x) = -x^2+4$  

\item $F(x) =-2(x-1.5)^2+4.5$

\setcounter{HW}{\value{enumi}}
\end{enumerate}
\end{multicols}

\begin{multicols}{2}
\begin{enumerate}
\setcounter{enumi}{\value{HW}}

\item $(-\infty, -3] \cup [1, \infty)$

\item  $\left(-\infty, -\frac{1}{4}\right) \cup \left(-\frac{1}{4}, \infty \right)$

\setcounter{HW}{\value{enumi}}
\end{enumerate}
\end{multicols}

\begin{multicols}{2}
\begin{enumerate}
\setcounter{enumi}{\value{HW}}

\item  No solution
\item  $(-\infty, \infty)$


\setcounter{HW}{\value{enumi}}
\end{enumerate}
\end{multicols}

\begin{multicols}{2}
\begin{enumerate}
\setcounter{enumi}{\value{HW}}

\item  $\left\{2 \right\}$
\item No solution


\setcounter{HW}{\value{enumi}}
\end{enumerate}
\end{multicols}

\begin{multicols}{2}
\begin{enumerate}
\setcounter{enumi}{\value{HW}}

\item $\left[-\frac{1}{3}, 4 \right]$
\item $(0, 1)$

\setcounter{HW}{\value{enumi}}
\end{enumerate}
\end{multicols}

\begin{multicols}{2}
\begin{enumerate}
\setcounter{enumi}{\value{HW}}


\item  $\left(-\infty, 1-\frac{\sqrt{6}}{2} \right) \cup \left(1+\frac{\sqrt{6}}{2}, \infty \right)$

\item  $\left(-\infty, \frac{5 - \sqrt{73}}{6} \right] \cup \left[\frac{5 + \sqrt{73}}{6}, \infty \right)$


\setcounter{HW}{\value{enumi}}
\end{enumerate}
\end{multicols}

\begin{multicols}{2}
\begin{enumerate}
\setcounter{enumi}{\value{HW}}

\item {\scriptsize $\left(-3\sqrt{2}, -\sqrt{11} \right] \cup \left[-\sqrt{7}, 0 \right) \cup \left(0, \sqrt{7} \right] \cup \left[\sqrt{11}, 3\sqrt{2} \right)$}
\item $\left[-2-\sqrt{7}, -2+\sqrt{7} \right] \cup [1, 3]$


\setcounter{HW}{\value{enumi}}
\end{enumerate}
\end{multicols}



\begin{multicols}{2}
\begin{enumerate}
\setcounter{enumi}{\value{HW}}

\item $(-\infty, \infty)$
\item  $(-\infty, -1] \cup \left\{ 0 \right\} \cup [1,\infty)$

\setcounter{HW}{\value{enumi}}
\end{enumerate}
\end{multicols}


\begin{multicols}{2}
\begin{enumerate}
\setcounter{enumi}{\value{HW}}


\item  $[-6,-3] \cup [-2, \infty)$

\item  $(-\infty, 1) \cup \left(2, \frac{3+\sqrt{17}}{2}\right)$


\setcounter{HW}{\value{enumi}}
\end{enumerate}
\end{multicols}


\begin{enumerate}
\setcounter{enumi}{\value{HW}}

\item \begin{itemize}

\item $P(x) = -2x^2+28x-26$, for $0 \leq x \leq 15$.

\item $7$ T-shirts should be made and  sold to maximize profit. 

\item The maximum profit is $\$72$. 

\item The price per T-shirt should be set at $\$16$ to maximize profit. 

\item The break even points are $x=1$ and $x=13$, so to make a profit, between 1 and 13 T-shirts need to be made and sold.

\end{itemize}

\item  \begin{itemize}

\item   $P(x) = -x^2+25x-100$, for $0 \leq x \leq 35$

\item  Since the vertex occurs at $x=12.5$, and it is impossible to make or sell $12.5$ bottles of tonic, maximum profit occurs when either $12$ or $13$ bottles of tonic are made and sold.

\item  The maximum profit is $\$56$.

\item  The price per bottle can be either $\$23$ (to sell 12 bottles) or $\$22$ (to sell 13 bottles.)  Both will result in the maximum profit.

\item The break even points are $x=5$ and $x=20$, so to make a profit, between 5 and 20 bottles of tonic need to be made and sold.

\end{itemize}



\item \begin{itemize}

\item  $P(x) = -3x^2+72x-240$, for $0 \leq x \leq 30$

\item  $12$ cups of lemonade need to be made and sold to maximize profit.

\item  The maximum profit is $192$\textcent \, or $\$1.92$.

\item  The price per cup should be set at $54$\textcent \, per cup to maximize profit.

\item  The break even points are $x=4$ and $x=20$, so to make a profit, between 4 and 20 cups of lemonade need to be made and sold.


\end{itemize}


\item \begin{itemize}

\item $P(x) = -0.5 x^2+9x-36$, for $0 \leq x \leq 24$

\item  $9$ pies should be made and sold to maximize the daily profit.

\item The maximum daily profit is $\$4.50$.

\item  The price per pie should be set at $\$7.50$ to maximize profit.

\item  The break even points are $x=6$ and $x=12$, so to make a profit, between 6 and 12 pies  need to be made and sold daily.

\end{itemize}

\item \begin{itemize}

\item  $P(x) = -2x^2+120x-1000$, for $0 \leq x \leq 70$

\item  $30$ scooters need to be made and sold to maximize profit.

\item  The maximum monthly profit is $800$ hundred dollars, or $\$80,\!000$.

\item The price per scooter should be set at $80$ hundred dollars, or $\$8000$ per scooter.

\item  The break even points are $x=10$ and $x=50$, so to make a profit, between 10 and 50 scooters  need to be made and sold monthly.

\end{itemize}

\setcounter{HW}{\value{enumi}}
\end{enumerate}

\begin{enumerate}
\setcounter{enumi}{\value{HW}}


\item 495 cookies

\item The vertex is (approximately) $(29.60, 22.66)$, which corresponds to a maximum fuel economy of 22.66 miles per gallon, reached sometime between 2009 and 2010 (29 -- 30 years after 1980.)  Unfortunately, the model is only valid up until 2008 (28 years after 1908.)  So, at this point, we are using the model to \textit{predict} the maximum fuel economy.



\item  $64^{\circ}$ at 2 PM (8 hours after 6 AM.)

\item  5000 pens should be produced for a cost of $\$200$.

\item 8 feet by 16 feet; maximum area is 128 square feet.

\item 50 feet by 50 feet;  maximum area is 2500 feet;  he can raise 100 average alpacas. 

\item The largest rectangle has area $12.25$ square inches.


\item  $2$ seconds.


\item  The rocket reaches its maximum height of $500$ feet $10$ seconds after lift-off.


\item  The hammer reaches a maximum height of approximately $13.62$ feet.  The hammer is in the air approximately $1.61$ seconds.  

\setcounter{HW}{\value{enumi}}
\end{enumerate}


\begin{enumerate}
\setcounter{enumi}{\value{HW}}


\item \begin{enumerate}

\item The applied domain is $[0, \infty)$.

\addtocounter{enumii}{2}

\item The height function is this case is $s(t) = -4.9t^{2} + 15t$.  The vertex of this parabola is approximately $(1.53, 11.48)$ so the maximum height reached by the marble is $11.48$ meters.  It hits the ground again when $t \approx 3.06$ seconds.

\item The revised height function is $s(t) = -4.9t^{2} + 15t + 25$ which has zeros at $t \approx -1.20$ and $t \approx 4.26$.  We ignore the negative value and claim that the marble will hit the ground after $4.26$ seconds.

\item Shooting down means the initial velocity is negative so the height functions becomes $s(t) = -4.9t^{2} - 15t + 25$.

\end{enumerate}

\item Make the vertex of the parabola $(0, 10)$ so that the point on the top of the left-hand tower where the cable connects is $(-200, 100)$ and the point on the top of the right-hand tower is $(200, 100)$.  Then the parabola is given by $p(x) = \frac{9}{4000}x^{2} + 10$.  Standing $50$ feet to the right of the left-hand tower means you're standing at $x= -150$ and $p(-150) = 60.625$.  So the cable is 60.625 feet above the bridge deck there.


\setcounter{HW}{\value{enumi}}
\end{enumerate}

\begin{enumerate}
\setcounter{enumi}{\value{HW}}

\item \begin{enumerate}

\item The line for the Thursday data is $y = -.12x + 237.69$.  We have $r = -.9568$ and $r^{2} = .9155$ so this is a really good fit.

\item The line for the Saturday data is $y = -0.000693x + 235.94$.  We have $r = -0.008986$ and $r^{2} = 0.0000807$ which is horrible.  This data is not even close to linear.  

\item The parabola for the Saturday data is $y = 0.003x^{2} - 0.21x + 238.30$.  We have $R^{2} = .47497$ which isn't good.  Thus the data isn't modeled well by a quadratic function, either.

\item The Thursday linear model had my weight on January 1, 2010 at 193.77 pounds.  The Saturday models give 235.69 and 563.31 pounds, respectively.  The Thursday line has my weight going below 0 pounds in about five and a half years, so that's no good.  The quadratic has a positive leading coefficient which would mean unbounded weight gain for the rest of my life.  The Saturday line, which mathematically does not fit the data at all, yields a plausible weight prediction in the end.  I think this is why grown-ups talk about ``Lies, Damned Lies and Statistics.''

\end{enumerate}

\item \begin{enumerate}

\item The quadratic model for the cats in Portage county is $y = 1917803.54x^{2} - 16036408.29x + 24094857.7$.  Although $R^{2} = .70888$ this is not a good model because it's so far off for small values of $x$.  The model gives us 24,094,858 cats when $x = 0$ but we know $N(0) = 2$.

\item The quadratic model for the hours of daylight in Fairbanks, Alaska is $y = .51x^{2} + 6.23x - .36$.  Even with $R^{2} = .92295$ we should be wary of making predictions beyond the data.  Case in point, the model gives $-4.84$ hours of daylight when $x = 13$.  So January 21, 2010 will be ``extra dark''?  Obviously a parabola pointing down isn't telling us the whole story.

\end{enumerate}

\setcounter{HW}{\value{enumi}}
\end{enumerate}

\begin{multicols}{2}
\begin{enumerate}
\setcounter{enumi}{\value{HW}}
\addtocounter{enumi}{1}
\item $y = |1 -x^{2}|$

\begin{mfpic}[12]{-3}{3}{-1}{8}
\axes
\tlabel[cc](3,-0.5){\scriptsize $x$}
\tlabel[cc](0.5,8){\scriptsize $y$}
\penwd{1.25pt}
\arrow \reverse \function{-3,-1,0.1}{x**2 - 1}
\function{-1,1,0.1}{1-x**2}
\arrow \function{1,3,0.1}{x**2-1}
\point[4pt]{(-1,0),(1,0),(0,1)}
\xmarks{-2 step 1 until 2}
\ymarks{1,2,3,4,5,6,7}
\scriptsize
\tlpointsep{4pt}
\axislabels {x}{{$-2 \hspace{6pt}$} -2, {$-1 \hspace{6pt}$} -1, {$1$} 1, {$2$} 2}
\axislabels {y}{{$1$} 1, {$2$} 2, {$3$} 3, {$4$} 4, {$5$} 5, {$6$} 6, {$7$} 7}
\normalsize
\end{mfpic}

\item $\left(\dfrac{3 - \sqrt{7}}{2}, \dfrac{-1 + \sqrt{7}}{2} \right)$, $\left(\dfrac{3 + \sqrt{7}}{2}, \dfrac{-1 - \sqrt{7}}{2} \right)$

\setcounter{HW}{\value{enumi}}
\end{enumerate}
\end{multicols}

\begin{enumerate}
\setcounter{enumi}{\value{HW}}


\item $D(x) = x^2 + (2x+1)^2 = 5x^2+4x+1$ is minimized when $x=-\frac{2}{5}$.  Hence to find the  point on $y=2x+1$ closest to $(0,0)$ we substitute $x = -\frac{2}{5}$ into  $y=2x+1$ to get $\left(-\frac{2}{5}, \frac{1}{5}\right)$.

\setcounter{HW}{\value{enumi}}
\end{enumerate}

\begin{multicols}{3}
\begin{enumerate}
\setcounter{enumi}{\value{HW}}
\addtocounter{enumi}{2}

\item $x = \pm y\sqrt{10}$ \vphantom{$\dfrac{m \pm \sqrt{m^{2} + 4}}{2}$}
\item $x = \pm (y - 2) $ \vphantom{$\dfrac{m \pm \sqrt{m^{2} + 4}}{2}$}
\item $x = \dfrac{m \pm \sqrt{m^{2} + 4}}{2}$

\setcounter{HW}{\value{enumi}}
\end{enumerate}
\end{multicols}

\begin{multicols}{3}
\begin{enumerate}
\setcounter{enumi}{\value{HW}}


\item $y = \dfrac{3 \pm \sqrt{16x + 9}}{2}$ \vphantom{$\dfrac{m \pm \sqrt{m^{2} + 4}}{2g}$}
\item $y = 2 \pm x$ \vphantom{$\dfrac{m \pm \sqrt{m^{2} + 4}}{2g}$}
\item $t = \dfrac{v_{\mbox{\scriptsize $0$}} \pm \sqrt{v_{\mbox{\scriptsize $0$}}^{2} + 4gs_{\mbox{\scriptsize $0$}}}}{2g}  $

\setcounter{HW}{\value{enumi}}
\end{enumerate}
\end{multicols}

\begin{enumerate}
\setcounter{enumi}{\value{HW}}

\item
\begin{enumerate}

\item \begin{multicols}{3}

\begin{enumerate}

\item   $L(x) = x^2$

\item   $L(x) = 2x^2+x$

\item  $L(x) = -x^2+5x+1$

\end{enumerate}

\end{multicols}

\addtocounter{enumii}{1}

\vspace{-.1in}

\item The three points lie on the same line and we get $L(x) = -x+5$.

\item  To obtain a quadratic function, we require that the points are not collinear (i.e., they do not all lie on the same line.)


\end{enumerate}

\setcounter{HW}{\value{enumi}}
\end{enumerate}


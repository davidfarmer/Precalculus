\label{ExercisesforParabolas}

In Exercises \ref{parabolasketchfirst} - \ref{parabolasketchlast},  graph of the given equations in the $xy$-plane.  Find the vertex, focus and directrix.  Include the endpoints of the latus rectum in your sketch.

\begin{multicols}{2}
\begin{enumerate}

\item $(x - 3)^{2} = -16y$ \label{parabolasketchfirst}
\item $\left(x + \frac{7}{3}\right)^{2} = 2\left(y + \frac{5}{2}\right)$


\setcounter{HW}{\value{enumi}}
\end{enumerate}
\end{multicols}

\begin{multicols}{2}
\begin{enumerate}
\setcounter{enumi}{\value{HW}}


\item \label{paranotfcnone} $(y - 2)^{2} = -12(x + 3)$ 
\item  \label{paranotfcntwo} $(y + 4)^{2} = 4x$

\setcounter{HW}{\value{enumi}}
\end{enumerate}
\end{multicols}

\begin{multicols}{2}
\begin{enumerate}
\setcounter{enumi}{\value{HW}}


\item $(x-1)^2 = 4(y+3)$
\item $(x+2)^2 = -20(y-5)$


\setcounter{HW}{\value{enumi}}
\end{enumerate}
\end{multicols}

\begin{multicols}{2}
\begin{enumerate}
\setcounter{enumi}{\value{HW}}

\item \label{paranotfcnthree} $(y-4)^2 = 18(x-2)$
\item  \label{paranotfcnfour} $\left(y+ \frac{3}{2}\right)^2 = -7 \left(x+ \frac{9}{2}\right)$ \label{parabolasketchlast}


\setcounter{HW}{\value{enumi}}
\end{enumerate}
\end{multicols}


In Exercises \ref{stdfrmparabolafirst} - \ref{stdfrmparabolalast}, put the equation into standard form.  Find the vertex, focus and directrix.\footnote{\ldots assuming the equation were graphed in the $xy$-plane.}

\begin{multicols}{2}
\begin{enumerate}
\setcounter{enumi}{\value{HW}}

\item  \label{paranotfcnfive} $y^{2} - 10y - 27x + 133 = 0$ \label{stdfrmparabolafirst}
\item $25x^{2} + 20x + 5y - 1 = 0$

\setcounter{HW}{\value{enumi}}
\end{enumerate}
\end{multicols}

\begin{multicols}{2}
\begin{enumerate}
\setcounter{enumi}{\value{HW}}

\item $x^2 + 2x - 8y + 49 = 0$
\item  \label{paranotfcnsix} $2y^2 + 4y +x - 8 = 0$

\setcounter{HW}{\value{enumi}}
\end{enumerate}
\end{multicols}

\begin{multicols}{2}
\begin{enumerate}
\setcounter{enumi}{\value{HW}}

\item $x^2-10x+12y+1=0$
\item   $3y^2-27y+4x+\frac{211}{4} = 0$ \label{stdfrmparabolalast}   \label{paranotfcnseven} 

\setcounter{HW}{\value{enumi}}
\end{enumerate}
\end{multicols}

\begin{enumerate}
\setcounter{enumi}{\value{HW}}


\item For each of the equations given in Exercises \ref{parabolasketchfirst} - \ref{stdfrmparabolalast} that do \textbf{not} describe $y$ as a function of $x$, find two or more explicit functions of $x$ represented by each of the equations.  (See Example \ref{horizontalparabolaex}.)


\setcounter{HW}{\value{enumi}}
\end{enumerate}

In Exercises \ref{buildparafromgraphfirst} - \ref{buildparafromgraphlast}, find an equation for the parabola whose graph is given.

\begin{multicols}{2}
\begin{enumerate}
\setcounter{enumi}{\value{HW}}

\item $~$ \label{buildparafromgraphfirst}

\begin{mfpic}[13]{-5}{5}{-5}{5}
\axes
\tlabel[cc](5,-0.5){\scriptsize $x$}
\tlabel[cc](0.5,5){\scriptsize $y$}
\tlabel[cc](1, 1){\scriptsize $(0,2)$}
\tlabel[cc](-2.25,-3.75){\scriptsize $(-2,-6)$}
\xmarks{-4,-3,-2,-1,1,2,3,4}
\ymarks{-4,-3,-2, -1, 1,2,3,4}
\tlpointsep{4pt}
\scriptsize
\axislabels {x}{ {$-4 \hspace{7pt}$} -4, {$-3 \hspace{7pt}$} -3, {$-2 \hspace{7pt}$} -2, {$-1 \hspace{7pt}$} -1, {$1$} 1, {$2$} 2, {$3$} 3, {$4$} 4}
\axislabels {y}{ 1, {$4$} 2, {$6$} 3, {$8$} 4, {$-4$} -2, {$-6$} -3}
\penwd{1.25pt}
\arrow \reverse \arrow \function{-4.8,0.8,0.1}{(x+2)**2-3}
\point[4pt]{(-2,-3), (0,1)}
\normalsize
\end{mfpic} 

\vfill

\columnbreak

\item $~$

\begin{mfpic}[13]{-5}{5}{-5}{5}
\axes
\tlabel[cc](5,-0.5){\scriptsize $x$}
\tlabel[cc](0.5,5){\scriptsize $y$}
\tlabel[cc](-1.25, 4.25){\scriptsize $(0,4)$}
\tlabel[cc](3,0.75){\scriptsize $(\sqrt{2},0)$}
\tlabel[cc](-3.25,0.75){\scriptsize $(-\sqrt{2},0)$}
\xmarks{-3,,-1,1,3}
\ymarks{-4,-3,-2, -1, 1,2,3,4}
\tlpointsep{4pt}
\scriptsize
\axislabels {x}{ {$-2 \hspace{7pt}$} -3, {$-1 \hspace{7pt}$} -1, {$1$} 1, {$2$} 3}
\axislabels {y}{{$-1$} -1,{$1$} 1, {$2$} 2, {$3$} 3,  {$-2$} -2, {$-3$} -3, {$-4$} -4}
\penwd{1.25pt}
\arrow \reverse \arrow \function{-3,3,0.1}{4-(x**2)}
\point[4pt]{(0,4), (-2,0), (2,0)}
\normalsize
\end{mfpic} 

\setcounter{HW}{\value{enumi}}
\end{enumerate}
\end{multicols}


\pagebreak

\begin{multicols}{2}
\begin{enumerate}
\setcounter{enumi}{\value{HW}}


\item $~$   

\begin{mfpic}[13]{-6}{5}{-3}{7}
\axes
\tlabel[cc](5,-0.5){\scriptsize $x$}
\tlabel[cc](0.5,7){\scriptsize $y$}
\tlabel[cc](-5.5, 2){\scriptsize $(-4,2)$}
\tlabel[cc](1, 4.75){\scriptsize $(0,4)$}
\tlabel[cc](1, -1){\scriptsize $(0, 0)$}
%\tlabel[cc](-0.5,-1){\scriptsize $\left(0, \frac{1}{2} \right)$}
\xmarks{-4,-3,-2,-1,1,2,3,4}
\ymarks{-2 step 1 until 6}
\tlpointsep{4pt}
\scriptsize
\axislabels {x}{ {$-4 \hspace{7pt}$} -4, {$-3 \hspace{7pt}$} -3, {$-2 \hspace{7pt}$} -2, {$-1 \hspace{7pt}$} -1,  {$4$} 4}
\axislabels {y}{{$1$} 1, {$2$} 2, {$3$} 3,  {$5$} 5,{$6$} 6, {$-1$} -1, {$-2$} -2}
\penwd{1.25pt}
\arrow  \function{-4,5,0.1}{2-sqrt(x+4)}
\arrow  \function{-4,5,0.1}{2+sqrt(x+4)}
\point[4pt]{(-4,2), (0,0), (0,4)}
%\tcaption{ \scriptsize $x$,$y$-intercept $(0,0)$}
\normalsize
\end{mfpic} 

\vfill

\columnbreak

\item $~$ \label{buildparafromgraphlast} 

\begin{mfpic}[13]{-5}{5}{-5}{5}
\axes
\tlabel[cc](5,-0.5){\scriptsize $x$}
\tlabel[cc](0.5,5){\scriptsize $y$}
\tlabel[cc](1.25,2){\scriptsize $(0,4)$}
\tlabel[cc](1.5,-2){\scriptsize $(0,-4)$}
\tlabel[cc](2,-0.75){\scriptsize $(1,0)$}
%\tlabel[cc](-1.5, 0.5){\scriptsize $(-1,0)$}
%\tlabel[cc](-0.5,-1){\scriptsize $\left(0, \frac{1}{2} \right)$}
\xmarks{-4,-3,-2,-1,1,2,3,4}
\ymarks{-4 step 1 until 4}
\tlpointsep{4pt}
\scriptsize
\axislabels {x}{ {$-4 \hspace{7pt}$} -4, {$-3 \hspace{7pt}$} -3, {$-2 \hspace{7pt}$} -2, {$-1 \hspace{7pt}$} -1,   {$4$} 4}
\axislabels {y}{{$2$} 1, {$4$} 2, {$6$} 3, {$8$} 4, {$-2$} -1, {$-4$} -2, {$-6$} -3, {$-8$} -4}
\penwd{1.25pt}
\arrow \reverse \function{-5,1,0.1}{2*sqrt(1-x)}
\arrow \reverse \function{-5,1,0.1}{-2*sqrt(1-x)}
\point[4pt]{(1,0), (0,2), (0,-2)}
%\tcaption{ \scriptsize $x$-intercept $(1,0)$, $y$-intercept $(0,2)$}
\normalsize
\end{mfpic} 


\setcounter{HW}{\value{enumi}}
\end{enumerate}
\end{multicols}


In Exercises \ref{buildparafirst} - \ref{buildparalast}, find an equation for the parabola which fits the given criteria.

\begin{multicols}{2}
\begin{enumerate}
\setcounter{enumi}{\value{HW}}

\item Vertex $(7, 0)$, focus $(0, 0)$. \label{buildparafirst}
\item Focus $(10, 1)$, directrix $x = 5$.


\setcounter{HW}{\value{enumi}}
\end{enumerate}
\end{multicols}

\begin{enumerate}
\setcounter{enumi}{\value{HW}}


\item Vertex $(-8, -9)$; $(0, 0)$ and $(-16, 0)$ are points on the curve.
\item The endpoints of latus rectum are $(-2, -7)$ and $(4, -7)$.\label{buildparalast}

\setcounter{HW}{\value{enumi}}
\end{enumerate}





\begin{enumerate}
\setcounter{enumi}{\value{HW}}

\item  The mirror in Carl's flashlight is a paraboloid of revolution.  If the mirror is 5 centimeters in diameter and 2.5 centimeters deep, where should the light bulb be placed so it is at the focus of the mirror?

\item  A parabolic Wi-Fi antenna is constructed by taking a flat sheet of metal and bending it into a parabolic shape.\footnote{This shape is called a `parabolic cylinder.'}  If the cross section of the antenna is a parabola which is 45 centimeters wide and 25 centimeters deep, where should the receiver be placed to maximize reception?

\item  \label{parabolaarch} A parabolic arch is constructed which is 6 feet wide at the base and 9 feet tall in the middle. Find the height of the arch exactly 1 foot in from the base of the arch. 

\item  A popular novelty item is the `mirage bowl.'  Follow this  \href{http://spie.org/etop/2007/etop07methodsV.pdf}{\underline{link}} to see another startling application of the reflective property of the parabola.

\item With the help of your classmates, research spinning liquid mirrors.  To get you started,  \href{http://www.astro.ubc.ca/LMT/lzt/}{\underline{here}}.

\end{enumerate}

\newpage

\subsection{Answers}

\begin{enumerate}

\item \begin{multicols}{2}
{\small $(x - 3)^{2} = -16y$}\\
{\small Vertex $(3, 0)$}\\
{\small Focus $(3, -4)$}\\
{\small Directrix $y = 4$}\\
{\small Endpoints of latus rectum $(-5, -4)$, $(11, -4)$}\\

\vfill

\columnbreak

\begin{mfpic}[10]{-6}{12}{-5}{5}
\axes
\xmarks{-5 step 1 until 11}
\ymarks{-4 step 1 until 4}
\arrow \reverse \arrow \polyline{(-6,4),(12,4)}
\plotsymbol[4pt]{Asterisk}{(3,-4)}
\tlabel(12,-0.5){\scriptsize $x$}
\tlabel(0.5,5){\scriptsize $y$}
\point[4pt]{(3,0),(-5,-4),(11,-4)}
\tlpointsep{4pt}
\tiny
\axislabels {x}{{$-5 \hspace{7pt}$} -5, {$-4 \hspace{7pt}$} -4, {$-3 \hspace{7pt}$} -3, {$-2 \hspace{7pt}$} -2, {$-1 \hspace{7pt}$} -1, {$1$} 1, {$2$} 2, {$3$} 3, {$4$} 4, {$5$} 5, {$6$} 6, {$7$} 7, {$8$} 8, {$9$} 9, {$10$} 10, {$11$} 11}
\axislabels {y}{{$-4$} -4, {$-3$} -3, {$-2$} -2, {$-1$} -1, {$1$} 1, {$2$} 2, {$3$} 3, {$4$} 4}
\normalsize
\penwd{1.25pt}
\arrow \reverse \arrow \function{-5.5,11.5,0.1}{((x - 3)**2)/(-16)}
\end{mfpic}
\end{multicols}

\smallskip

\item  \begin{multicols}{2}
{\small $\left(x + \frac{7}{3}\right)^{2} = 2\left(y + \frac{5}{2}\right)$}\\
{\small Vertex $\left(-\frac{7}{3}, -\frac{5}{2} \right)$}\\
{\small Focus $\left(-\frac{7}{3}, -2 \right)$}\\
{\small Directrix $y = -3$}\\
{\small Endpoints of latus rectum $\left(-\frac{10}{3}, -2 \right)$, $\left(-\frac{4}{3}, -2 \right)$}\\

\vfill

\columnbreak


\begin{mfpic}[15][20]{-6}{1}{-4}{3}
\axes
\xmarks{-5 step 1 until 0}
\ymarks{-3 step 1 until 2}
\arrow \reverse \arrow \polyline{(-5,-3),(1,-3)}
\plotsymbol[4pt]{Asterisk}{(-2.333,-2)}
\tlabel(1,-0.5){\scriptsize $x$}
\tlabel(0.5,3){\scriptsize $y$}
\point[4pt]{(-2.333,-2.5),(-3.333,-2),(-1.333,-2)}
\tlpointsep{4pt}
\tiny
\axislabels {x}{{$-5 \hspace{7pt}$} -5, {$-4 \hspace{7pt}$} -4, {$-3 \hspace{7pt}$} -3, {$-2 \hspace{7pt}$} -2, {$-1 \hspace{7pt}$} -1}
\axislabels {y}{{$-3$} -3, {$-2$} -2, {$-1$} -1, {$1$} 1, {$2$} 2}
\normalsize
\penwd{1.25pt}
\arrow \reverse \arrow \function{-5.5,0.8,0.1}{((x + (7/3))**2)/2 - (5/2)}
\end{mfpic}
\end{multicols}

\smallskip

\item \begin{multicols}{2} 

{\small $(y - 2)^{2} = -12(x + 3)$} \\
{\small Vertex $(-3, 2)$} \\
{\small Focus $(-6, 2)$} \\
{\small Directrix $x = 0$}\\
{\small Endpoints of latus rectum $(-6, 8)$, $(-6, -4)$}\\

\vfill

\columnbreak


\begin{mfpic}[10]{-8}{1}{-5}{9}
\axes
\xmarks{-7 step 1 until 0}
\ymarks{-4 step 1 until 8}
\plotsymbol[4pt]{Asterisk}{(-6,2)}
\tlabel(1,-0.5){\scriptsize $x$}
\tlabel(0.5,9){\scriptsize $y$}
\point[4pt]{(-3,2),(-6,-4),(-6,8)}
\tlpointsep{4pt}
\tiny
\axislabels {x}{{$-7 \hspace{7pt}$} -7, {$-6 \hspace{7pt}$} -6, {$-5 \hspace{7pt}$} -5, {$-4 \hspace{7pt}$} -4, {$-3 \hspace{7pt}$} -3, {$-2 \hspace{7pt}$} -2, {$-1 \hspace{7pt}$} -1}
\axislabels {y}{{$-4$} -4, {$-3$} -3, {$-2$} -2, {$-1$} -1, {$1$} 1, {$2$} 2, {$3$} 3, {$4$} 4, {$5$} 5, {$6$} 6, {$7$} 7, {$8$} 8}
\normalsize
\penwd{1.25pt}
\arrow \reverse \function{-6.8,-3,0.1}{2+sqrt((-12*x) - 36)}
\arrow \reverse \function{-6.8,-3,0.1}{2-sqrt((-12*x) - 36)}
\end{mfpic}
\end{multicols}

\pagebreak

\item \begin{multicols}{2} 
{\small $(y + 4)^{2} = 4x$}\\
{\small Vertex $(0,-4)$} \\
{\small Focus $(1,-4)$} \\
{\small Directrix $x = -1$}\\
{\small Endpoints of latus rectum $(1, -2)$, $(1, -6)$}\\

\vfill

\columnbreak


\begin{mfpic}[15]{-2}{5}{-9}{1}
\axes
\xmarks{-1 step 1 until 4}
\ymarks{-8 step 1 until 0}
\arrow \reverse \arrow \polyline{(-1,-9),(-1,1)}
\plotsymbol[4pt]{Asterisk}{(1,-4)}
\tlabel(5,-0.5){\scriptsize $x$}
\tlabel(0.5,1){\scriptsize $y$}
\point[4pt]{(0,-4),(1,-2),(1,-6)}
\tlpointsep{4pt}
\tiny
\axislabels {x}{{$-1 \hspace{7pt}$} -1, {$1$} 1, {$2$} 2, {$3$} 3, {$4$} 4}
\axislabels {y}{{$-8$} -8, {$-7$} -7, {$-6$} -6, {$-5$} -5, {$-4$} -4, {$-3$} -3, {$-2$} -2, {$-1$} -1}
\normalsize
\penwd{1.25pt}
\arrow \function{0,5,0.1}{-4-(2*sqrt(x))}
\arrow \function{0,5,0.1}{-4+(2*sqrt(x))}
\end{mfpic}
\end{multicols}

\smallskip


\item  \begin{multicols}{2}
{\small $(x-1)^2 = 4(y+3)$}\\
{\small Vertex $\left(1, -3\right)$}\\
{\small Focus $\left(1, -2 \right)$}\\
{\small Directrix $y = -4$}\\
{\small Endpoints of latus rectum $\left(3, -2 \right)$, $\left(-1, -2 \right)$}\\

\vfill

\columnbreak


\begin{mfpic}[15]{-4}{5}{-5}{1}
\axes
\xmarks{-3 step 1 until 4}
\ymarks{-4 step 1 until 0}
\arrow \reverse \arrow \polyline{(-5,-4),(5,-4)}
\plotsymbol[4pt]{Asterisk}{(1,-2)}
\tlabel(5,-0.5){\scriptsize $x$}
\tlabel(0.5,1){\scriptsize $y$}
\point[4pt]{(3,-2),(1,-3),(-1,-2)}
\tlpointsep{4pt}
\tiny
\axislabels {x}{{$-3 \hspace{7pt}$} -3, {$-2 \hspace{7pt}$} -2, {$-1 \hspace{7pt}$} -1, {$1$} 1, {$2$} 2, {$3$} 3, {$4$} 4}
\axislabels {y}{{$-4$} -4, {$-3$} -3, {$-2$} -2, {$-1$} -1}
\normalsize
\penwd{1.25pt}
\arrow \reverse \arrow \function{-3,5,0.1}{((x -1)**2)/4 - 3}
\end{mfpic}
\end{multicols}

\smallskip

\item \begin{multicols}{2}
{\small $(x+2)^2 = -20(y-5)$}\\
{\small Vertex $\left(-2, 5\right)$}\\
{\small Focus $\left(-2, 0 \right)$}\\
{\small Directrix $y = 10$}\\
{\small Endpoints of latus rectum $\left(-12, 0 \right)$, $\left(8, 0 \right)$}\\

\vfill

\columnbreak


\begin{mfpic}[7.5][10]{-13}{9}{-1}{11}
\axes
\xmarks{-12 step 1 until 8}
\ymarks{1 step 1 until 10}
\arrow \reverse \arrow \polyline{(-13,10),(9,10)}
\plotsymbol[4pt]{Asterisk}{(-2,0)}
\tlabel(9,-0.5){\scriptsize $x$}
\tlabel(0.5,11){\scriptsize $y$}
\point[4pt]{(-12,0),(-2,5),(8,0)}
\tlpointsep{4pt}
\tiny
\axislabels {x}{{$-12 \hspace{7pt}$} -12,  {$-10 \hspace{7pt}$} -10, {$-8 \hspace{7pt}$} -8, {$-6 \hspace{7pt}$} -6, {$-4 \hspace{7pt}$} -4,  {$-2 \hspace{7pt}$} -2, {$2$} 2,  {$4$} 4,  {$6$} 6, {$8$} 8}
\axislabels {y}{{$1$} 1, {$2$} 2, {$3$} 3, {$4$} 4, {$5$} 5, {$6$} 6, {$7$} 7, {$8$} 8, {$9$} 9, {$10$} 10}
\normalsize
\penwd{1.25pt}
\arrow \reverse \arrow \function{-13,9,0.1}{((x +2)**2)/(0-20) + 5}
\end{mfpic}
\end{multicols}

\smallskip


\item \begin{multicols}{2}
{\small $(y-4)^2 = 18(x-2)$}\\
{\small Vertex $\left(2, 4\right)$}\\
{\small Focus $\left( \frac{13}{2}, 4 \right)$}\\
{\small Directrix $x = -\frac{5}{2}$}\\
{\small Endpoints of latus rectum $\left(\frac{13}{2}, -5 \right)$, $\left(\frac{13}{2}, 13 \right)$}\\

\vfill

\columnbreak


\begin{mfpic}[15][7.5]{-3}{8}{-6}{14}
\axes
\xmarks{-2 step 1 until 7}
\ymarks{-5 step 1 until 13}
\arrow \reverse \arrow \polyline{(-2.5,-6),(-2.5,14)}
\plotsymbol[4pt]{Asterisk}{(6.5,4)}
\tlabel(8,-0.5){\scriptsize $x$}
\tlabel(0.5,14){\scriptsize $y$}
\point[4pt]{(6.5,-5),(2,4),(6.5,13)}
\tlpointsep{4pt}
\tiny
\axislabels {x}{{$-1$} -1, {$1$} 1,  {$2$} 2,  {$3$} 3, {$4$} 4, {$5$} 5,  {$6$} 6,  {$7$} 7}
\axislabels {y}{{$-5$} -5, {$-3$} -3, {$-1$} -1, {$1$} 1, {$3$} 3, {$5$} 5, {$7$} 7, {$9$} 9, {$11$} 11, {$13$} 13}
\normalsize
\penwd{1.25pt}
\arrow \function{2,8,0.1}{4+(sqrt(18*(x-2)))}
\arrow \function{2,8,0.1}{4-(sqrt(18*(x-2)))}
\end{mfpic}
\end{multicols}

\smallskip

\item \begin{multicols}{2} 
{\small $\left(y+ \frac{3}{2}\right)^2 = -7 \left(x+ \frac{9}{2}\right)$}\\
{\small Vertex $\left(-\frac{9}{2}, -\frac{3}{2}\right)$}\\
{\small Focus $\left( -\frac{25}{4}, -\frac{3}{2} \right)$}\\
{\small Directrix $x = -\frac{11}{4}$}\\
{\small Endpoints of latus rectum $\left(-\frac{25}{4}, 2 \right)$, $\left(-\frac{25}{4}, -5 \right)$}\\

\vfill

\columnbreak


\begin{mfpic}[15]{-7}{1}{-6}{3}
\axes
\xmarks{-6 step 1 until -1}
\ymarks{-5 step 1 until 2}
\arrow \reverse \arrow \polyline{(-2.75,-6),(-2.75,3)}
\plotsymbol[4pt]{Asterisk}{(-6.25,-1.5)}
\tlabel(1,-0.5){\scriptsize $x$}
\tlabel(0.5,3){\scriptsize $y$}
\point[4pt]{(-6.25,-5),(-4.5,-1.5),(-6.25,2)}
\tlpointsep{4pt}
\tiny
\axislabels {x}{{$-5 \hspace{7pt}$} -5,{$-4 \hspace{7pt}$} -4,{$-3 \hspace{7pt}$} -3,{$-2 \hspace{7pt}$} -2,{$-1 \hspace{7pt}$} -1}
\axislabels {y}{{$-5$} -5, {$-4$} -4, {$-3$} -3, {$-2$} -2, {$-1$} -1, {$1$} 1, {$2$} 2}
\normalsize
\penwd{1.25pt}
\arrow \function{-4.5,-7,0.1}{0-1.5+(sqrt((0-7)*(x+4.5)))}
\arrow \function{-4.5,-7,0.1}{0-1.5-(sqrt((0-7)*(x+4.5)))}
\end{mfpic}
\end{multicols}

\setcounter{HW}{\value{enumi}}
\end{enumerate}

\begin{multicols}{2}
\begin{enumerate}
\setcounter{enumi}{\value{HW}}


\item $(y - 5)^{2} = 27(x - 4)$\\
Vertex $(4, 5)$\\
Focus $\left( \frac{43}{4}, 5 \right)$\\
Directrix $x = -\frac{11}{4}$

\item $\left(x + \frac{2}{5} \right)^{2} = -\frac{1}{5}(y - 1)$\\
Vertex $\left( -\frac{2}{5}, 1 \right)$\\
Focus $\left( -\frac{2}{5}, \frac{19}{20} \right)$\\
Directrix $y = \frac{21}{20}$

\setcounter{HW}{\value{enumi}}
\end{enumerate}
\end{multicols}

\begin{multicols}{2}
\begin{enumerate}
\setcounter{enumi}{\value{HW}}


\item  $(x+1)^2=8(y-6)$ \\
Vertex $(-1,6)$\\
Focus $(-1,8)$ \\
Directrix $y=4$

\item  $(y+1)^2=-\frac{1}{2}(x-10)$\\
Vertex $(10,-1)$\\
Focus $\left(\frac{79}{8}, -1 \right)$\\
Directrix $x = \frac{81}{8}$

\setcounter{HW}{\value{enumi}}
\end{enumerate}
\end{multicols}

\begin{multicols}{2}
\begin{enumerate}
\setcounter{enumi}{\value{HW}}

\item $(x-5)^2 = -12(y-2)$\\
Vertex $(5,2)$\\
Focus $(5,-1)$ \\
Directrix $y=5$


\item  $\left(y-\frac{9}{2}\right)^2 = -\frac{4}{3} (x-2)$\\
Vertex $\left(2, \frac{9}{2}\right)$\\
Focus $\left(\frac{5}{3}, \frac{9}{2}\right)$\\
Directrix $x = \frac{7}{3}$

\setcounter{HW}{\value{enumi}}
\end{enumerate}
\end{multicols}

\begin{enumerate}
\setcounter{enumi}{\value{HW}}

\item  The equations which do not represent $y$ as a function of $x$ are:  \ref{paranotfcnone}, \ref{paranotfcntwo}, \ref{paranotfcnthree}, \ref{paranotfcnfour}, \ref{paranotfcnfive}, \ref{paranotfcnsix}, \ref{paranotfcnseven}.


For number \ref{paranotfcnone}:

\begin{itemize}

\item  $f(x) = 2+2 \sqrt{-3x-9}$ represents the upper half of the parabola.

\item  $g(x) =  2 - 2\sqrt{-3x-9}$ represents the lower half of the parabola.

\end{itemize}

For number \ref{paranotfcntwo}:

\begin{itemize}

\item  $f(x) = -4+2\sqrt{x}$ represents the upper half of the parabola.

\item  $g(x) =  -4 - 2\sqrt{x}$  represents the lower half of the parabola.

\end{itemize}


For number \ref{paranotfcnthree}:

\begin{itemize}

\item   $f(x) =4+3 \sqrt{2x-4}$ represents the upper half of the parabola.

\item  $g(x) =  4-3 \sqrt{2x-4}$  represents the lower half of the parabola.

\end{itemize}


For number \ref{paranotfcnfour}:

\begin{itemize}

\item   $f(x) =-\frac{3}{2} + \frac{1}{2} \sqrt{-28x - 126}$ represents the upper half of the parabola.

\item  $g(x) =-\frac{3}{2} -  \frac{1}{2} \sqrt{-28x - 126}$  represents the lower half of the parabola.

\end{itemize}


For number \ref{paranotfcnfive}:

\begin{itemize}

\item   $f(x) =5+3 \sqrt{3x-12}$ represents the upper half of the parabola.

\item  $g(x) =5- 3 \sqrt{3x-12}$ represents the lower half of the parabola.

\end{itemize}


For number \ref{paranotfcnsix}:

\begin{itemize}

\item   $f(x) =-1 + \frac{1}{2} \sqrt{-2x+20}$ represents the upper half of the parabola.

\item  $g(x) =-1 - \frac{1}{2} \sqrt{-2x+20}$ represents the lower half of the parabola.

\end{itemize}

For number \ref{paranotfcnseven}:

\begin{itemize}

\item   $f(x) = \frac{9}{2} + \frac{2}{3} \sqrt{-3x+6}$ represents the upper half of the parabola.

\item  $f(x) = \frac{9}{2} - \frac{2}{3} \sqrt{-3x+6}$ represents the lower half of the parabola.

\end{itemize}


\setcounter{HW}{\value{enumi}}
\end{enumerate}

\begin{multicols}{4}
\begin{enumerate}
\setcounter{enumi}{\value{HW}}

\item  $(x+2)^2 = \frac{1}{2} (y+6)$
\item  $x^2 = -\frac{1}{2}(y-4)$
\item $(y-2)^2=x+4$  \vphantom{$(x+2)^2 = \frac{1}{2} (y+6)$}
\item $y^2 = -16(x-1)$  \vphantom{$(x+2)^2 = \frac{1}{2} (y+6)$}


\setcounter{HW}{\value{enumi}}
\end{enumerate}
\end{multicols}


\begin{multicols}{3}
\begin{enumerate}
\setcounter{enumi}{\value{HW}}

\item $y^{2} = -28(x - 7)$
\item $(y - 1)^{2} = 10\left(x - \frac{15}{2} \right)$
\item $(x + 8)^{2} = \frac{64}{9}(y + 9)$

\setcounter{HW}{\value{enumi}}
\end{enumerate}
\end{multicols}

\begin{enumerate}
\setcounter{enumi}{\value{HW}}


\item $(x - 1)^{2} = 6\left(y + \frac{17}{2}\right)$ or $(x - 1)^{2} = -6\left(y + \frac{11}{2}\right)$

\setcounter{HW}{\value{enumi}}
\end{enumerate}





\begin{enumerate}
\setcounter{enumi}{\value{HW}}

\item  The bulb should be placed $0.625$ centimeters above the vertex of the mirror.\footnote{As verified by Carl himself!}

\item  The receiver should be placed $5.0625$ centimeters from the vertex of the cross section of the antenna.

\item  The arch can be modeled by $x^2=-(y-9)$ or $y=9-x^2$.  One foot in from the base of the arch corresponds to either $x = \pm 2$, so the height is $y=9-(\pm 2)^2=5$ feet.

\end{enumerate}
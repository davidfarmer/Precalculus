\label{ExercisesforPropertiesofLogarithms}

In Exercises \ref{expandlogfirst} - \ref{expandloglast}, expand the given logarithm and simplify.  Assume when necessary that all quantities represent positive real numbers.

\begin{multicols}{3}
\begin{enumerate}

\item $\ln(x^{3}y^{2})$ \vphantom{$\log_{2}\left(\dfrac{128}{x^{2} + 4}\right)$} \label{expandlogfirst}
\item $\log_{2}\left(\dfrac{128}{x^{2} + 4}\right)$
\item $\log_{5}\left(\dfrac{z}{25}\right)^{3}$ \vphantom{$\log_{2}\left(\dfrac{128}{x^{2} + 4}\right)$}

\setcounter{HW}{\value{enumi}}
\end{enumerate}
\end{multicols}

\begin{multicols}{3}
\begin{enumerate}
\setcounter{enumi}{\value{HW}}

\item $\log(1.23 \times 10^{37})$ \vphantom{$\ln\left(\dfrac{\sqrt{z}}{xy}\right)$}
\item $\ln\left(\dfrac{\sqrt{z}}{xy}\right)$
\item $\log_{5} \left(x^2 - 25 \right)$ \vphantom{$\ln\left(\dfrac{\sqrt{z}}{xy}\right)$}

\setcounter{HW}{\value{enumi}}
\end{enumerate}
\end{multicols}

\begin{multicols}{3}
\begin{enumerate}
\setcounter{enumi}{\value{HW}}

\item $\log_{\sqrt{2}} \left(4x^3\right)$
\item $\log_{\frac{1}{3}}(9x(y^{3} - 8))$
\item $\log\left(1000x^3y^5\right)$

\setcounter{HW}{\value{enumi}}
\end{enumerate}
\end{multicols}

\begin{multicols}{3}
\begin{enumerate}
\setcounter{enumi}{\value{HW}}

\item $\log_{3} \left(\dfrac{x^2}{81y^4}\right)$
\item $\ln\left(\sqrt[4]{\dfrac{xy}{ez}}\right)$
\item $\log_{6} \left(\dfrac{216}{x^3y}\right)^4$

\setcounter{HW}{\value{enumi}}
\end{enumerate}
\end{multicols}

\begin{multicols}{3}
\begin{enumerate}
\setcounter{enumi}{\value{HW}}

\item $\log\left(\dfrac{100x\sqrt{y}}{\sqrt[3]{10}}\right)$ \vphantom{$\log_{\frac{1}{2}}\left(\dfrac{4\sqrt[3]{x^2}}{y\sqrt{z}}\right)$}
\item $\log_{\frac{1}{2}}\left(\dfrac{4\sqrt[3]{x^2}}{y\sqrt{z}}\right)$
\item $\ln \left(\dfrac{\sqrt[3]{x}}{10 \sqrt{yz}}\right)$ \vphantom{$\log_{\frac{1}{2}}\left(\dfrac{4\sqrt[3]{x^2}}{y\sqrt{z}}\right)$} \label{expandloglast}

\setcounter{HW}{\value{enumi}}
\end{enumerate}
\end{multicols}

In Exercises \ref{combinelogfirst} - \ref{combineloglast}, use the properties of logarithms to write the expression as a single logarithm.

\begin{multicols}{2}
\begin{enumerate}
\setcounter{enumi}{\value{HW}}

\item $4\ln(x) + 2\ln(y)$ \label{combinelogfirst}
\item $\log_{2}(x) + \log_{2}(y) - \log_{2}(z)$

\setcounter{HW}{\value{enumi}}
\end{enumerate}
\end{multicols}

\begin{multicols}{2}
\begin{enumerate}
\setcounter{enumi}{\value{HW}}

\item $\log_{3}(x) - 2 \log_{3}(y)$
\item $\frac{1}{2}\log_{3}(x) - 2\log_{3}(y) - \log_{3}(z)$

\setcounter{HW}{\value{enumi}}
\end{enumerate}
\end{multicols}

\begin{multicols}{2}
\begin{enumerate}
\setcounter{enumi}{\value{HW}}
\item $2 \ln(x) -3 \ln(y) - 4\ln(z)$
\item $\log(x) - \frac{1}{3} \log(z) + \frac{1}{2} \log(y)$

\setcounter{HW}{\value{enumi}}
\end{enumerate}
\end{multicols}

\begin{multicols}{2}
\begin{enumerate}
\setcounter{enumi}{\value{HW}}

\item $-\frac{1}{3} \ln(x) - \frac{1}{3}\ln(y) + \frac{1}{3} \ln(z)$
\item $\log_{5}(x) - 3$

\setcounter{HW}{\value{enumi}}
\end{enumerate}
\end{multicols}

\begin{multicols}{2}
\begin{enumerate}
\setcounter{enumi}{\value{HW}}

\item $3 - \log(x)$
\item $\log_{7}(x) + \log_{7}(x - 3) - 2$

\setcounter{HW}{\value{enumi}}
\end{enumerate}
\end{multicols}

\begin{multicols}{2}
\begin{enumerate}
\setcounter{enumi}{\value{HW}}

\item $\ln(x) + \frac{1}{2}$ 
\item $\log_{2}(x) + \log_{4}(x)$ 

\setcounter{HW}{\value{enumi}}
\end{enumerate}
\end{multicols}

\begin{multicols}{2}
\begin{enumerate}
\setcounter{enumi}{\value{HW}}

\item $\log_{2}(x) + \log_{4}(x-1)$
\item $\log_{2}(x) + \log_{\frac{1}{2}}(x - 1)$ \label{combineloglast}

\setcounter{HW}{\value{enumi}}
\end{enumerate}
\end{multicols}


In Exercises \ref{changeofbasefirst} - \ref{changeofbaselast}, use the appropriate change of base formula to convert the given expression to an expression with the indicated base. 

\begin{multicols}{2}
\begin{enumerate}
\setcounter{enumi}{\value{HW}}

\item $7^{x - 1}$ to base $e$ \label{changeofbasefirst}
\item $\log_{3}(x + 2)$ to base 10

\setcounter{HW}{\value{enumi}}
\end{enumerate}
\end{multicols}

\begin{multicols}{2}
\begin{enumerate}
\setcounter{enumi}{\value{HW}}

\item $\left(\dfrac{2}{3}\right)^{x}$ to base $e$
\item $\log(x^{2} + 1)$ to base $e$ \vphantom{$\left(\dfrac{2}{3}\right)^{x}$}\label{changeofbaselast}

\setcounter{HW}{\value{enumi}}
\end{enumerate}
\end{multicols}

\newpage

In Exercises \ref{changeofbaseapproxfirst} - \ref{changeofbaseapproxlast}, use the appropriate change of base formula to approximate the logarithm.

\begin{multicols}{3}
\begin{enumerate}
\setcounter{enumi}{\value{HW}}

\item $\log_{3}(12)$ \label{changeofbaseapproxfirst}
\item $\log_{5}(80)$
\item $\log_{6}(72)$

\setcounter{HW}{\value{enumi}}
\end{enumerate}
\end{multicols}

\begin{multicols}{3}
\begin{enumerate}
\setcounter{enumi}{\value{HW}}

\item $\log_{4}\left(\dfrac{1}{10}\right)$
\item $\log_{\frac{3}{5}}(1000)$ \vphantom{$\log_{4}\left(\dfrac{1}{10}\right)$}
\item $\log_{\frac{2}{3}}(50)$ \vphantom{$\log_{4}\left(\dfrac{1}{10}\right)$} \label{changeofbaseapproxlast}

\setcounter{HW}{\value{enumi}}
\end{enumerate}
\end{multicols}

\begin{enumerate}
\setcounter{enumi}{\value{HW}}

\item \label{morethanoneforlogexercise} In Example \ref{intrologex} number \ref{findformulaforlogexample} in Section \ref{LogarithmicFunctions}, we obtained the solution  $F(x) = \log_{2}(-x+4)-3$ as one formula for the given graph by making a simplifying assumption that $b = -1$.  This exercises explores if there are any other solutions for different choices of $b$.

\begin{enumerate}

\item  Show  $G(x) =\log_{2}(-2x+8) - 4$ also fits the data for the given graph.

\item  Use properties of logarithms to show $G(x) = \log_{2}(-2x+8) -4  = \log_{2}(-x+4)-3 = F(x)$.

\item  With help from your classmates, find solutions to Example \ref{intrologex} number \ref{findformulaforlogexample} in Section \ref{LogarithmicFunctions} by assuming $b = -4$ and $b = -8$.  In each case, use properties of logarithms to show the solutions reduce to $F(x) = \log_{2}(-x+4)-3$.

\item  Using properties of logarithms and the fact that the range of $\log_{2}(x)$ is all real numbers, show that any function of the form $f(x) = a \log_{2}(bx-h) + k$ where $a \neq 0$ can be rewritten as: 

\[ f(x) = a \left( \log_{2}(bx-h) +  \frac{k}{a}\right) = a ( \log_{2}(bx -h) + \log_{2}(p)) = a \log_{2}(p(bx-h)) = a \log_{2}(pbx - ph),\]

where $\frac{k}{a} = \log_{2}(p)$ for some positive real number $p$. Relabeling, we get every function of the form $f(x) = a \log_{2}(bx-h) + k$ with four parameters ($a$, $b$, $h$, and $k$) can be rewritten as $f(x) = a \log_{2}(Bx - H)$, a formula with just three parameters: $a$, $B$, and $H$.

\smallskip

Show \textit{every} solution to Example \ref{intrologex} number \ref{findformulaforlogexample} in Section \ref{LogarithmicFunctions} can be written in the form  $f(x) = \log_{2}\left( -\frac{1}{8}  x + \frac{1}{2} \right)$ and that, in particular,  $F(x) = \log_{2}(-x+4) -3 = \log_{2}\left( -\frac{1}{8}  x + \frac{1}{2} \right) = f(x)$.  Hence, there is really just one solution to Example \ref{intrologex} number \ref{findformulaforlogexample} in Section \ref{LogarithmicFunctions}.

\end{enumerate}   

\item \label{HendersonHasselbalch} \index{Henderson-Hasselbalch Equation} The Henderson-Hasselbalch Equation:  Suppose $HA$ represents a weak acid. Then we have a reversible chemical reaction 
\[HA \rightleftharpoons H^{+} + A^{-}.\]  
The acid disassociation constant, $K_{a}$, is given by 
\[K_{a} = \frac{[H^{+}][A^{-}]}{[HA]} = [H^{+}]\frac{[A^{-}]}{[HA]},\]
where the square brackets denote the concentrations just as they did in Exercise \ref{pHexercise} in Section \ref{LogarithmicFunctions}.  The symbol p$K_{a}$ is defined similarly to pH in that p$K_{a} = -\log(K_{a})$.  Using the definition of pH from Exercise \ref{pHexercise} and the properties of logarithms, derive the Henderson-Hasselbalch Equation:

\[\mbox{pH} = \mbox{p}K_{a} + \log\dfrac{[A^{-}]}{[HA]}\]

\item Compare and contrast the graphs of $y = \ln(x^{2})$ and $y = 2\ln(x)$.

\item Prove the Quotient Rule and Power Rule for Logarithms.

\item Give numerical examples to show that, in general,

\begin{multicols}{2}
\begin{enumerate}

\item $\log_{b}(x + y) \neq \log_{b}(x) + \log_{b}(y)$
\item $\log_{b}(x - y) \neq \log_{b}(x) - \log_{b}(y)$
\setcounter{HWindent}{\value{enumii}}
\end{enumerate}
\end{multicols}

\begin{enumerate}
\setcounter{enumii}{\value{HWindent}}

\item $\log_{b}\left(\dfrac{x}{y}\right) \neq \dfrac{\log_{b}(x)}{\log_{b}(y)}$

\end{enumerate}

\item Research the history of logarithms including the origin of the word `logarithm' itself.  Why is the abbreviation of natural log `ln' and not `nl'?

\item There is a scene in the movie `Apollo 13' in which several people at Mission Control use slide rules to verify a computation.  Was that scene accurate?  Look for other pop culture references to logarithms and slide rules.


\setcounter{HW}{\value{enumi}}
\end{enumerate}


\newpage

\subsection{Answers}


\begin{multicols}{2}
\begin{enumerate}

\item $3\ln(x) + 2\ln(y)$
\item $7 - \log_{2}(x^{2} + 4)$

\setcounter{HW}{\value{enumi}}
\end{enumerate}
\end{multicols}

\begin{multicols}{2}
\begin{enumerate}
\setcounter{enumi}{\value{HW}}


\item $3\log_{5}(z) - 6$
\item $\log(1.23) + 37$

\setcounter{HW}{\value{enumi}}
\end{enumerate}
\end{multicols}

\begin{multicols}{2}
\begin{enumerate}
\setcounter{enumi}{\value{HW}}

\item $\frac{1}{2}\ln(z) - \ln(x) - \ln(y)$
\item  $\log_{5}(x-5) + \log_{5}(x+5)$

\setcounter{HW}{\value{enumi}}
\end{enumerate}
\end{multicols}

\begin{multicols}{2}
\begin{enumerate}
\setcounter{enumi}{\value{HW}}

\item  $3\log_{\sqrt{2}}(x) + 4$
\item \small$-2 + \log_{\frac{1}{3}}(x) + \log_{\frac{1}{3}}(y - 2) + \log_{\frac{1}{3}}(y^{2} + 2y + 4)$\normalsize

\setcounter{HW}{\value{enumi}}
\end{enumerate}
\end{multicols}

\begin{multicols}{2}
\begin{enumerate}
\setcounter{enumi}{\value{HW}}

\item $3 + 3\log(x) + 5 \log(y)$
\item $2\log_{3}(x) - 4 - 4\log_{3}(y)$

\setcounter{HW}{\value{enumi}}
\end{enumerate}
\end{multicols}

\begin{multicols}{2}
\begin{enumerate}
\setcounter{enumi}{\value{HW}}

\item $\frac{1}{4} \ln(x) + \frac{1}{4} \ln(y) - \frac{1}{4} - \frac{1}{4} \ln(z)$
\item $12-12\log_{6}(x) - 4\log_{6}(y)$

\setcounter{HW}{\value{enumi}}
\end{enumerate}
\end{multicols}

\begin{multicols}{2}
\begin{enumerate}
\setcounter{enumi}{\value{HW}}

\item $\frac{5}{3}+\log(x)+\frac{1}{2}\log(y)$
\item $-2+\frac{2}{3}\log_{\frac{1}{2}}(x)-\log_{\frac{1}{2}}(y)-\frac{1}{2}\log_{\frac{1}{2}}(z)$

\setcounter{HW}{\value{enumi}}
\end{enumerate}
\end{multicols}

\begin{multicols}{2}
\begin{enumerate}
\setcounter{enumi}{\value{HW}}

\item $\frac{1}{3} \ln(x) - \ln(10) - \frac{1}{2}\ln(y)-\frac{1}{2}\ln(z)$
\item $\ln(x^{4}y^{2})$
\setcounter{HW}{\value{enumi}}
\end{enumerate}
\end{multicols}

\begin{multicols}{3}
\begin{enumerate}
\setcounter{enumi}{\value{HW}}

\item $\log_{2}\left(\frac{xy}{z}\right)$
\item $\log_{3} \left( \frac{x}{y^2} \right)$
\item $\log_{3}\left(\frac{\sqrt{x}}{y^{2}z}\right)$

\setcounter{HW}{\value{enumi}}
\end{enumerate}
\end{multicols}

\begin{multicols}{3}
\begin{enumerate}
\setcounter{enumi}{\value{HW}}


\item $\ln\left( \frac{x^2}{y^3z^4} \right)$
\item $\log\left(\frac{x \sqrt{y}}{\sqrt[3]{z}}  \right)$
\item $\ln\left(\sqrt[3]{\frac{z}{xy}}   \right)$

\setcounter{HW}{\value{enumi}}
\end{enumerate}
\end{multicols}

\begin{multicols}{3}
\begin{enumerate}
\setcounter{enumi}{\value{HW}}

\item $\log_{5}\left(\frac{x}{125}\right)$
\item $\log\left(\frac{1000}{x}\right)$
\item $\log_{7}\left(\frac{x(x - 3)}{49}\right)$

\setcounter{HW}{\value{enumi}}
\end{enumerate}
\end{multicols}

\begin{multicols}{3}
\begin{enumerate}
\setcounter{enumi}{\value{HW}}

\item $\ln \left(x \sqrt{e} \right)$
\item $\log_{2}\left(x^{3/2}\right)$
\item $\log_{2}\left(x \sqrt{x-1}\right)$


\setcounter{HW}{\value{enumi}}
\end{enumerate}
\end{multicols}


\begin{multicols}{3}
\begin{enumerate}
\setcounter{enumi}{\value{HW}}
\item $\vphantom{\frac{\log(x + 2)}{\log(3)}}\log_{2}\left(\frac{x}{x - 1}\right)$ 
\item $\vphantom{\frac{\log(x + 2)}{\log(3)}}7^{x - 1} = e^{(x - 1)\ln(7)}$
\item $\log_{3}(x + 2) = \frac{\log(x + 2)}{\log(3)}$


\setcounter{HW}{\value{enumi}}
\end{enumerate}
\end{multicols}


\begin{multicols}{2}
\begin{enumerate}
\setcounter{enumi}{\value{HW}}

\item $\left(\frac{2}{3}\right)^{x} = e^{x\ln(\frac{2}{3})}$
\item $\log(x^{2} + 1) = \frac{\ln(x^{2} + 1)}{\ln(10)}$

\setcounter{HW}{\value{enumi}}
\end{enumerate}
\end{multicols}

\begin{multicols}{2}
\begin{enumerate}
\setcounter{enumi}{\value{HW}}

\item $\log_{3}(12) \approx 2.26186$
\item $\log_{5}(80) \approx 2.72271$

\setcounter{HW}{\value{enumi}}
\end{enumerate}
\end{multicols}

\begin{multicols}{2}
\begin{enumerate}
\setcounter{enumi}{\value{HW}}

\item $\log_{6}(72) \approx 2.38685$
\item $\log_{4}\left(\frac{1}{10}\right) \approx -1.66096$

\setcounter{HW}{\value{enumi}}
\end{enumerate}
\end{multicols}

\begin{multicols}{2}
\begin{enumerate}
\setcounter{enumi}{\value{HW}}
\item $\log_{\frac{3}{5}}(1000) \approx -13.52273$
\item $\log_{\frac{2}{3}}(50) \approx -9.64824$

\setcounter{HW}{\value{enumi}}
\end{enumerate}
\end{multicols}

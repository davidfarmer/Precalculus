\label{ExercisesforInverseFunctions}

In Exercises \ref{verifyinversehwfirst} - \ref{verifyinversehwlast}, verify the given pairs of functions are inverses algebraically and graphically.  

\begin{multicols}{2}
\begin{enumerate}

\item $f(x) = 2x+7$ and $g(x) = \dfrac{x-7}{2}$ \label{verifyinversehwfirst}
\item $f(x) = \dfrac{5-3x}{4}$ and $g(x) = -\dfrac{4}{3} x + \dfrac{5}{3}$.


\setcounter{HW}{\value{enumi}}
\end{enumerate}
\end{multicols}


\begin{multicols}{2}
\begin{enumerate}
\setcounter{enumi}{\value{HW}}

\item $f(t) = \dfrac{5}{t-1}$ and $g(t) = \dfrac{t+5}{t}$ 
\item \label{owninverseexample} $f(t)  = \dfrac{t}{t-1}$ and $g(t) = f(t) =  \dfrac{t}{t-1}$


\setcounter{HW}{\value{enumi}}
\end{enumerate}
\end{multicols}


\begin{multicols}{2}
\begin{enumerate}
\setcounter{enumi}{\value{HW}}

\item $f(x) = \sqrt{4-x}$ and $g(x) = -x^2+4$, $x \geq 0$
\item $f(x) = 1-\sqrt{x+1}$ and $g(x) = x^2-2x$, $x \leq 1$.

\setcounter{HW}{\value{enumi}}
\end{enumerate}
\end{multicols}

\begin{multicols}{2}
\begin{enumerate}
\setcounter{enumi}{\value{HW}}

\item $f(t) = (t-1)^3+5$ and $g(t) = \sqrt[3]{t-5}+1$
\item  $f(t) = -\sqrt[4]{t-2}$ and $g(t) = t^4+2$, $t \leq 0$.  \label{verifyinversehwlast}


\setcounter{HW}{\value{enumi}}
\end{enumerate}
\end{multicols}


In Exercises \ref{inversehwfirst} - \ref{inversehwlast}, show that the given function is one-to-one and find its inverse.  Check your answers algebraically and graphically.  Verify the range of the function is the domain of its inverse and vice-versa.

\begin{multicols}{2}
\begin{enumerate}
\setcounter{enumi}{\value{HW}}

\item $f(x) = 6x - 2$ \label{inversehwfirst}
\item $f(x) = 42-x$


\setcounter{HW}{\value{enumi}}
\end{enumerate}
\end{multicols}


\begin{multicols}{2}
\begin{enumerate}
\setcounter{enumi}{\value{HW}}

\item $g(t) = \dfrac{t-2}{3} + 4$
\item $g(t)  = 1 - \dfrac{4+3t}{5}$


\setcounter{HW}{\value{enumi}}
\end{enumerate}
\end{multicols}


\begin{multicols}{2}
\begin{enumerate}
\setcounter{enumi}{\value{HW}}

\item $f(x) = \sqrt{3x-1}+5$
\item $f(x) = 2-\sqrt{x - 5}$

\setcounter{HW}{\value{enumi}}
\end{enumerate}
\end{multicols}

\begin{multicols}{2}
\begin{enumerate}
\setcounter{enumi}{\value{HW}}

\item $g(t) = 3\sqrt{t-1}-4$
\item $g(t) = 1 - 2\sqrt{2t+5}$


\setcounter{HW}{\value{enumi}}
\end{enumerate}
\end{multicols}

\begin{multicols}{2}
\begin{enumerate}
\setcounter{enumi}{\value{HW}}

\item $f(x) = \sqrt[5]{3x-1}$
\item $f(x) = 3-\sqrt[3]{x-2}$

\setcounter{HW}{\value{enumi}}
\end{enumerate}
\end{multicols}

\begin{multicols}{2}
\begin{enumerate}
\setcounter{enumi}{\value{HW}}

\item $g(t) = t^2 - 10t$, $t \geq 5$
\item $g(t) = 3(t + 4)^{2} - 5, \; t \leq -4$

\setcounter{HW}{\value{enumi}}
\end{enumerate}
\end{multicols}


\begin{multicols}{2}
\begin{enumerate}
\setcounter{enumi}{\value{HW}}

\item $f(x) = x^2-6x+5, \; x \leq 3$
\item $f(x) = 4x^2 + 4x + 1$, $x < -1$

\setcounter{HW}{\value{enumi}}
\end{enumerate}
\end{multicols}


\begin{multicols}{2}
\begin{enumerate}
\setcounter{enumi}{\value{HW}}

\item $g(t) = \dfrac{3}{4-t}$
\item $g(t) = \dfrac{t}{1-3t}$

\setcounter{HW}{\value{enumi}}
\end{enumerate}
\end{multicols}


\begin{multicols}{2}
\begin{enumerate}
\setcounter{enumi}{\value{HW}}

\item $f(x) = \dfrac{2x-1}{3x+4}$
\item $f(x) = \dfrac{4x + 2}{3x - 6}$

\setcounter{HW}{\value{enumi}}
\end{enumerate}
\end{multicols}


\begin{multicols}{2}
\begin{enumerate}
\setcounter{enumi}{\value{HW}}

\item $g(t) = \dfrac{-3t - 2}{t + 3}$ 

\item $g(t) = \dfrac{t-2}{2t-1}$  \label{inversehwlast}

\setcounter{HW}{\value{enumi}}
\end{enumerate}
\end{multicols}

\begin{enumerate}
\setcounter{enumi}{\value{HW}}

\item  Explain why each set of ordered pairs  below represents a  one-to-one function and find the inverse.


\begin{enumerate}

\item  $F = \{ (0,0), (1,1), (2,-1), (3,2), (4,-2), (5,3), (6,-3)  \}$

\item  $G = \{ (0,0), (1,1), (2,-1), (3,2), (4,-2), (5,3), (6,-3), \ldots \}$  

NOTE:  The difference between $F$ and $G$ is the  `$\ldots$.'  

\item  $P = \{ (2t^5, 3t-1) \, | \, \text{$t$ is a real number.} \}$

\item  $Q = \{ (n, n^2) \, | \, \text{$n$ is a \textit{natural} number.} \}$\footnote{Recall this means $n = 0, 1, 2, \ldots$.}

\end{enumerate}

\setcounter{HW}{\value{enumi}}
\end{enumerate}

\newpage

In Exercises \ref{inversefromgraphfirst} - \ref{inversefromgraphlast}, explain why each graph  represents\footnote{or, more precisely, \textit{appears} to represent \ldots}  a one-to-one function and graph its inverse.


\begin{multicols}{2}
\begin{enumerate}
\setcounter{enumi}{\value{HW}}

\item $y = f(x)$  \label{inversefromgraphfirst}

\begin{mfpic}[18]{-3}{3}{-0.5}{5.5}
\axes
\tlabel[cc](3,-0.5){\scriptsize $x$}
\tlabel[cc](0.5,5.5){\scriptsize $y$}
\xmarks{-2, -1, 0, 1, 2}
\ymarks{ 0, 1, 2, 3,4,5}
\tcaption{Asymptote: $y = 0$.}
\tlpointsep{4pt}
\scriptsize
\tlabel[cc](1, 1){$(0,1)$}
\tlabel[cc](2, 2){$(1,2)$}
\tlabel[cc](2.75, 4){$(2,4)$}
\axislabels {x}{{\scriptsize $-2 \hspace{7pt}$} -2,{\scriptsize $-1 \hspace{7pt}$} -1,{$1$} 1, {$2$} 2}
\axislabels {y}{{$2$} 2,{$3$} 3,{$4$} 4,{$5$} 5}
\normalsize
\penwd{1.25pt}
\arrow \reverse \arrow \function{-2.5, 2.5, 0.1}{2**x}
\point[4pt]{(0,1), (1,2), (2,4)}
\end{mfpic}




\item  $y = g(t)$ 

\begin{mfpic}[18][9]{-4}{3}{-6}{6}
\axes
\tlabel[cc](3,-0.5){\scriptsize $t$}
\tlabel[cc](0.5,6){\scriptsize $y$}
\xmarks{ -3,-2,-1,1,2}
\ymarks{-5,-4,-3,-2,-1,1,2,3,4,5}
\tcaption{Asymptote: $t=2$.}
\tlpointsep{4pt}
\scriptsize
\dashed \polyline{(2,-6), (2,6)}
\gclear \tlabelrect(2, 0.5){$(1,0)$}
\tlabel[cc](1, 2){$(0,2)$}
\tlabel[cc](-1.75, 5){$(-2,4)$}
\axislabels {x}{{\scriptsize $-3 \hspace{7pt}$} -3,{\scriptsize $-2 \hspace{7pt}$} -2,{\scriptsize $-1 \hspace{7pt}$} -1,  {$2$} 2 }
\axislabels {y}{{$-2$} -2, {$-3$} -3,{$-4$} -4,  {$-5$} -5,  {$1$} 1,  {$3$} 3, {$5$} 5, {$4$} 4}
\normalsize
\penwd{1.25pt}
\arrow \reverse \arrow \parafcn{-2.5, 2.5, 0.1}{(2-2**t,2*t)}
\point[4pt]{(1,0), (0,2), (-2,4)}
\end{mfpic}

\setcounter{HW}{\value{enumi}}
\end{enumerate}
\end{multicols}

\begin{multicols}{2}
\begin{enumerate}
\setcounter{enumi}{\value{HW}}

\item $y = S(t) $

\begin{mfpic}[15]{-5}{5}{-4}{4}
\axes
\tlabel[cc](5,-0.25){\scriptsize $t$}
\tlabel[cc](0.25,4){\scriptsize $y$}
\tlabel[cc](-4,-3.5){\scriptsize $(-4,-3)$}
\tlabel[cc](0.75,-0.5){\scriptsize $(0,0)$}
\tlabel[cc](4,3.5){\scriptsize $(4,3)$}
\xmarks{-4,-3,-2,-1,1,2,3,4}
\ymarks{-3,-2,-1,1,2,3}
\tlpointsep{5pt}
\tcaption{Domain: $[-4,4]$.}
\scriptsize
\axislabels {x}{{$-4 \hspace{7pt}$} -4,{$-3 \hspace{7pt}$} -3,{$-2 \hspace{7pt}$} -2,{$-1 \hspace{7pt}$} -1,{$2$} 2,{$3$} 3,{$4$} 4}
\axislabels {y}{{$-3$} -3,{$-2$} -2, {$-1$} -1, {$1$} 1, {$2$} 2, {$3$} 3}
\normalsize
\penwd{1.25pt}
\function{-4,4,0.1}{3*sin(1.570796327*x/4)}
\point[4pt]{(-4,-3), (0,0), (4,3)}
\end{mfpic} 

\columnbreak

\item  $y = R(s)$ \label{inversefromgraphlast}

\begin{mfpic}[15]{-5}{5}{-4}{4}
\axes
\tlabel[cc](5,-0.25){\scriptsize $s$}
\tlabel[cc](0.25,4){\scriptsize $y$}
\tlabel[cc](-2,-1.5){\scriptsize $\left(-\frac{1}{2},-\frac{3}{2} \right)$}
\tlabel[cc](-0.75,0.5){\scriptsize $(0,0)$}
\tlabel[cc](1.75,1.5){\scriptsize $\left(\frac{1}{2},\frac{3}{2} \right)$}
%\tlabel[cc](3, 3.5){\scriptsize asymptote $y=3$}
%\tlabel[cc](-2.75,-3.5){\scriptsize asymptote $y=-3$}
\tcaption{Asymptotes: $y = \pm 3$.}
\xmarks{-4,-3,-2,-1,1,2,3,4}
\ymarks{-3,-2,-1,1,2,3}
\tlpointsep{5pt}
\scriptsize
%\axislabels {x}{{$-4 \hspace{7pt}$} -4,{$-3 \hspace{7pt}$} -3, {$-1 \hspace{7pt}$} -1,{$1$} 1,{$3$} 3,{$4$} 4}
%\axislabels {y}{{$-4$} -4,{$-3$} -3,{$-2$} -2, {$-1$} -1, {$1$} 1, {$2$} 2, {$3$} 3, {$4$} 4}
\normalsize
\dashed \polyline {(-5,3), (5,3)}
\dashed \polyline {(-5,-3), (5,-3)}
\penwd{1.25pt}
\arrow \reverse \arrow \parafcn{-2.8,2.8,0.1}{( 0.5*(tan( 0.5236*t))   ,   t   )}
\point[4pt]{(0,0), (0.5,1.5), (-0.5,-1.5)}
\end{mfpic} 

\setcounter{HW}{\value{enumi}}
\end{enumerate}
\end{multicols}


\begin{enumerate}
\setcounter{enumi}{\value{HW}}

\item  The price of a dOpi media player, in dollars per dOpi, is given as a function of the weekly sales $x$ according to the formula $p(x) = 450-15x$ for $0 \leq x \leq 30$.

\begin{enumerate}

\item  Find $p^{-1}(x)$ and state its domain.

\item  Find and interpret $p^{-1}(105)$.

\item  The profit (in dollars) made from producing and selling $x$ dOpis per week is given by the formula $P(x)= -15x^2+350x-2000$, for $0 \leq x \leq 30$.  Find $\left(P \circ p^{-1}\right)(x)$ and determine what price per dOpi would yield the maximum profit.  What is the maximum profit?  How many dOpis need to be produced and sold to achieve the maximum profit?
\end{enumerate}

\item Show that the Fahrenheit to Celsius conversion function found in Exercise \ref{celsiustofahr} in Section \ref{LinearFunctions} is invertible and that its inverse is the Celsius to Fahrenheit conversion function.

\item Analytically show that the function $f(x) = x^3 + 3x + 1$ is one-to-one.  Use Theorem \ref{inversefunctionprops} to help you compute $f^{-1}(1), \; f^{-1}(5), \;$ and $f^{-1}(-3)$.  What happens when you attempt to find a formula for $f^{-1}(x)$?


\item  Let $f(x) = \dfrac{2x}{x^2-1}$.  

\begin{enumerate}

\item  Graph $y = f(x)$ using the techniques  in Section \ref{RationalGraphs}.  Check your answer using a graphing utility.

\item Verify that $f$ is one-to-one on the interval $(-1,1)$.  

\item Use the procedure outlined on Page \pageref{inverseprocedure} to find the formula for $f^{-1}(x)$ for $-1 < x < 1$.

\item  Since $f(0) = 0$, it should be the case that $f^{-1}(0) = 0$.  What goes wrong when you attempt to substitute $x=0$ into $f^{-1}(x)$?  Discuss with your classmates how this problem arose and possible remedies.

\end{enumerate}

\item With the help of your classmates, explain why a function which is either strictly increasing or strictly decreasing on its entire domain would have to be one-to-one, hence invertible.

\item If $f$ is odd and invertible, prove that $f^{-1}$ is also odd.

\item \label{fcircginverse} Let $f$ and $g$ be invertible functions.  With the help of your classmates show that $(f \circ g)$ is one-to-one, hence invertible, and that $(f \circ g)^{-1}(x) = (g^{-1} \circ f^{-1})(x)$.

\setcounter{HW}{\value{enumi}}
\end{enumerate}

With help from your classmates, find the inverses of the functions in Exercises \ref{genericinversefirst} - \ref{genericinverselast}.

\begin{multicols}{2}
\begin{enumerate}
\setcounter{enumi}{\value{HW}}

\item $f(x) = ax + b, \; a \neq 0$ \label{genericinversefirst}
\item $f(x) = a\sqrt{x - h} + k, \; a \neq 0, x \geq h$


\setcounter{HW}{\value{enumi}}
\end{enumerate}
\end{multicols}

\begin{multicols}{2}
\begin{enumerate}
\setcounter{enumi}{\value{HW}}
\item $f(x) = ax^{2} + bx + c$ where $a \neq 0, \, x \geq -\dfrac{b}{2a}$.

\item $f(x) = \dfrac{ax + b}{cx + d},\;$ (See Exercise \ref{whatconditions} below.) \label{genericinverselast}

\setcounter{HW}{\value{enumi}}
\end{enumerate}
\end{multicols}

\begin{enumerate}
\setcounter{enumi}{\value{HW}}

\item \label{whatconditions} What conditions must you place on the values of $a, b, c$ and $d$ in Exercise \ref{genericinverselast} in order to guarantee that the function is invertible?


\item  The function given in number \ref{owninverseexample} is an example of a function which is its own inverse.  

\begin{enumerate}

\item Algebraically verify every function of the form: $f(x) = \dfrac{ax + b}{cx - a}$ is its own inverse.  

What assumptions do you need to make about the values of  $a$, $b$, and $c$?

\item  Under what conditions is $f(x) = mx + b$, $m \neq 0$ its own inverse?  Prove your answer.

\end{enumerate}
\setcounter{HW}{\value{enumi}}
\end{enumerate}

\newpage

\subsection{Answers}

\begin{multicols}{2}
\begin{enumerate}
\addtocounter{enumi}{8}

\item $f^{-1}(x) = \dfrac{x + 2}{6}$
\item $f^{-1}(x) = 42-x$

\setcounter{HW}{\value{enumi}}
\end{enumerate}
\end{multicols}

\begin{multicols}{2}
\begin{enumerate}
\setcounter{enumi}{\value{HW}}

\item  $g^{-1}(t) = 3t-10$
\item $g^{-1}(t)  = -\frac{5}{3} t + \frac{1}{3}$


\setcounter{HW}{\value{enumi}}
\end{enumerate}
\end{multicols}

\begin{multicols}{2}
\begin{enumerate}
\setcounter{enumi}{\value{HW}}


\item $f^{-1}(x) = \frac{1}{3}(x-5)^2+\frac{1}{3}$, $x \geq 5$
\item $f^{-1}(x) = (x - 2)^{2} + 5, \; x \leq 2$

\setcounter{HW}{\value{enumi}}
\end{enumerate}
\end{multicols}

\begin{multicols}{2}
\begin{enumerate}
\setcounter{enumi}{\value{HW}}


\item $g^{-1}(t) = \frac{1}{9}(t+4)^2+1$, $t \geq -4$

\item $g^{-1}(t) = \frac{1}{8}(t-1)^2-\frac{5}{2}$, $t \leq 1$

\setcounter{HW}{\value{enumi}}
\end{enumerate}
\end{multicols}

\begin{multicols}{2}
\begin{enumerate}
\setcounter{enumi}{\value{HW}}

\item $f^{-1}(x) = \frac{1}{3} x^{5} + \frac{1}{3}$
\item $f^{-1}(x) = -(x-3)^3+2$

\setcounter{HW}{\value{enumi}}
\end{enumerate}
\end{multicols}

\begin{multicols}{2}
\begin{enumerate}
\setcounter{enumi}{\value{HW}}

\item $g^{-1}(t) = 5 + \sqrt{t+25}$
\item $g^{-1}(t) = -\sqrt{\frac{t + 5}{3}} - 4$

\setcounter{HW}{\value{enumi}}
\end{enumerate}
\end{multicols}

\begin{multicols}{2}
\begin{enumerate}
\setcounter{enumi}{\value{HW}}

\item $f^{-1}(x) = 3 - \sqrt{x+4}$
\item $f^{-1}(x) =-\frac{\sqrt{x}+1}{2}$, $x > 1$

\setcounter{HW}{\value{enumi}}
\end{enumerate}
\end{multicols}

\begin{multicols}{2}
\begin{enumerate}
\setcounter{enumi}{\value{HW}}

\item $g^{-1}(t) = \dfrac{4t-3}{t}$
\item $g^{-1}(t) = \dfrac{t}{3t+1}$

\setcounter{HW}{\value{enumi}}
\end{enumerate}
\end{multicols}

\begin{multicols}{2}
\begin{enumerate}
\setcounter{enumi}{\value{HW}}

\item $f^{-1}(x) = \dfrac{4x+1}{2-3x}$
\item $f^{-1}(x) = \dfrac{6x + 2}{3x - 4}$

\setcounter{HW}{\value{enumi}}
\end{enumerate}
\end{multicols}

\begin{multicols}{2}
\begin{enumerate}
\setcounter{enumi}{\value{HW}}

\item $g^{-1}(t) = \dfrac{-3t - 2}{t + 3}$
\item $g^{-1}(t) = \dfrac{t-2}{2t-1}$ 

\setcounter{HW}{\value{enumi}}
\end{enumerate}
\end{multicols}

\begin{enumerate}
\setcounter{enumi}{\value{HW}}

\item

\begin{enumerate}

\item  None of the first coordinates of the ordered pairs in $F$ are repeated, so $F$ is a function and none of the second coordinates of the ordered pairs of $F$ are repeated, so $F$ is one-to-one.   $F^{-1} = \{ (0,0), (1,1), (-1,2), (2,3), (-2,4), (3,5), (-3,6)  \}$

\item  Because of the `$\ldots$' it is helpful to determine a formula for the matching. For the even numbers $n$, $n = 0, 2, 4, \ldots$, the ordered pair $\left(n, -\frac{n}{2} \right)$ is in $G$.  For the odd numbers  $n = 1, 3, 5, \ldots$, the ordered pair $\left(n, \frac{n+1}{2} \right)$ is in $G$.  Hence, given any input to $G$, $n$, whether it be even or odd, there is only one output from $G$, either $-\frac{n}{2}$ or $\frac{n+1}{2}$, both of which are functions of $n$. To show $G$ is one to one, we note that if the output from $G$ is $0$ or less, then it must be of the form $-\frac{n}{2}$ for an even number $n$.  Moreover, if $-\frac{n}{2} = -\frac{m}{2}$, then $n = m$. In the case we are looking at outputs from $G$ which are greater than $0$, then it must be of the form $\frac{n+1}{2}$ for an odd number $n$.  In this, too, if  $\frac{n+1}{2} = \frac{m+1}{2}$, then $n = m$.  Hence, in any case, if the outputs from $G$ are the same, then the inputs to $G$ had to be the same so  $G$ is one-to-one and $G^{-1} = \{ (0,0), (1,1), (-1,2), (2,3), (-2,4), (3,5), (-3,6), \ldots \}$  


\item  To show $P$ is a function we note that if we have the same inputs to $P$, say $2t^{5} = 2u^{5}$, then $t = u$.  Hence the corresponding outputs, $2t-1$ and $3u-1$, are equal, too. To show $P$ is one-to-one, we note that if we have the same outputs from $P$, $3t-1 = 3u-1$, then $t = u$.  Hence, the corresponding  inputs $2t^5$  and $2u^5$ are equal, too. Hence $P$ is one-to-one and $P^{-1} = \{ (3t-1, 2t^5) \, | \, \text{$t$ is a real number.} \}$

\item  To show $Q$ is a function, we note that if we have the same inputs to $Q$, say $n = m$, then the outputs from $Q$, namely $n^2$ and $m^2$ are equal. To show $Q$ is one-to-one, we note that if we get the same output from $Q$, namely $n^2 = m^2$, then $n = \pm m$.  However since $n$ and $m$ are \textit{natural} numbers, both $n$ and $m$ are positive so $n = m$. Hence $Q$ is one-to-one and $Q^{-1} = \{ (n^2, n) \, | \, \text{$n$ is a \textit{natural} number.} \}$.

\end{enumerate}

\setcounter{HW}{\value{enumi}}
\end{enumerate}


\begin{multicols}{2}
\begin{enumerate}
\setcounter{enumi}{\value{HW}}

\item $y = f^{-1}(x)$. Asymptote: $x = 0$.

\begin{mfpic}[18]{-0.5}{5.5}{-3.5}{3.5}
\axes
\tlabel[cc](5.5,-0.5){\scriptsize $x$}
\tlabel[cc](0.5,3.5){\scriptsize $y$}
\ymarks{-2, -1, 0, 1, 2}
\xmarks{ 0, 1, 2, 3,4,5}
%\tcaption{Asymptote: $x = 0$.}
\tlpointsep{4pt}
\scriptsize
\tlabel[cc](1.5, -0.5){$(1,0)$}
\tlabel[cc](1.75, 1.5){$(2,1)$}
\tlabel[cc](4, 2.5){$(4,2)$}
\axislabels {y}{{\scriptsize $-2$} -2,{\scriptsize $-1$} -1,{$1$} 1, {$2$} 2}
\axislabels {x}{{$3$} 3,{$4$} 4,{$5$} 5}
\normalsize
\penwd{1.25pt}
\arrow \reverse \arrow \parafcn{-2.5, 2.5, 0.1}{(2**t, t)}
\point[4pt]{(1,0), (2,1), (4,2)}
\end{mfpic}




\item  $y = g^{-1}(t)$. Asymptote: $y=2$.

\begin{mfpic}[9][18]{-6}{6}{-4}{3}
\axes
\tlabel[cc](6,-0.5){\scriptsize $t$}
\tlabel[cc](0.5,3){\scriptsize $y$}
\ymarks{ -3,-2,-1,1,2}
\xmarks{-5,-4,-3,-2,-1,1,2,3,4,5}
%\tcaption{Asymptote: $y=2$.}
\tlpointsep{4pt}
\scriptsize
\dashed \polyline{(-6,2), (6,2)}
\tlabel[cc](2.5, 0.5){$(2,0)$}
\tlabel[cc](1,1.25){$(0,1)$}
\tlabel[cc](2.25, -2){$(4,-2)$}
\axislabels {y}{{\scriptsize $-3$} -3,{\scriptsize $-2$} -2,{\scriptsize $-1$} -1,  {$2$} 2 }
\axislabels {x}{{$-2 \hspace{7pt}$} -2, {$-3 \hspace{7pt}$} -3,{$-4 \hspace{7pt}$} -4,  {$-5\hspace{7pt}$} -5,  {$1$} 1,  {$3$} 3, {$5$} 5, {$4$} 4}
\normalsize
\penwd{1.25pt}
\arrow \reverse \arrow \parafcn{-2.5, 2.5, 0.1}{(2*t, 2-2**t)}
\point[4pt]{(0,1), (2,0), (4,-2)}
\end{mfpic}

\setcounter{HW}{\value{enumi}}
\end{enumerate}
\end{multicols}

\begin{multicols}{2}
\begin{enumerate}
\setcounter{enumi}{\value{HW}}

\item $y = S^{-1}(t)$. Domain $[-3,3]$.

\begin{mfpic}[15]{-4}{4}{-5}{5}
\axes
\tlabel[cc](4,-0.25){\scriptsize $t$}
\tlabel[cc](0.25,4){\scriptsize $y$}
\tlabel[cc](-3,-4.5){\scriptsize $(-3,-4)$}
\tlabel[cc](0.75,-0.5){\scriptsize $(0,0)$}
\tlabel[cc](3, 4.5){\scriptsize $(3,4)$}
\ymarks{-4,-3,-2,-1,1,2,3,4}
\xmarks{-3,-2,-1,1,2,3}
\tlpointsep{5pt}
%\tcaption{Domain: $[-3,3]$.}
\scriptsize
\axislabels {y}{{$-4$} -4,{$-3$} -3,{$-2$} -2,{$-1$} -1,{$2$} 2,{$3$} 3,{$4$} 4, {$1$} 1,}
\axislabels {x}{{$-3 \hspace{7pt}$} -3,{$-2 \hspace{7pt}$} -2, {$-1 \hspace{7pt}$} -1,  {$2$} 2, {$3$} 3}
\normalsize
\penwd{1.25pt}
\parafcn{-4,4,0.1}{(3*sin(1.570796327*t/4), t)}
\point[4pt]{(-3,-4), (0,0), (3,4)}
\end{mfpic} 

\columnbreak

\item  $y = R^{-1}(s)$.  Asymptotes: $s = \pm 3$.

\begin{mfpic}[15]{-4}{4}{-5}{5}
\axes
\tlabel[cc](4,-0.25){\scriptsize $s$}
\tlabel[cc](0.25,5){\scriptsize $y$}
\gclear\tlabelrect(-1,-1.5){\scriptsize $\left(-\frac{3}{2},-\frac{1}{2} \right)$}
\tlabel[cc](-0.75,0.5){\scriptsize $(0,0)$}
\tlabel[cc](1.5,1.5){\scriptsize $\left(\frac{3}{2},\frac{1}{2} \right)$}
%\tlabel[cc](3, 3.5){\scriptsize asymptote $y=3$}
%\tlabel[cc](-2.75,-3.5){\scriptsize asymptote $y=-3$}
%\tcaption{Asymptotes: $s = \pm 3$.}
\ymarks{-4,-3,1,2,3,4}
\xmarks{-3,-2,-1,1,2,3}
\tlpointsep{5pt}
\scriptsize
%\axislabels {x}{{$-4 \hspace{7pt}$} -4,{$-3 \hspace{7pt}$} -3, {$-1 \hspace{7pt}$} -1,{$1$} 1,{$3$} 3,{$4$} 4}
%\axislabels {y}{{$-4$} -4,{$-3$} -3,{$-2$} -2, {$-1$} -1, {$1$} 1, {$2$} 2, {$3$} 3, {$4$} 4}
\normalsize
\dashed \polyline {(3, -5), (3, 5)}
\dashed \polyline {(-3,-5), (-3,5)}
\penwd{1.25pt}
\arrow \reverse \arrow \parafcn{-2.8,2.8,0.1}{( t, 0.5*(tan( 0.5236*t)))}
\point[4pt]{(0,0), (1.5,0.5), (-1.5,-0.5)}
\end{mfpic} 

\setcounter{HW}{\value{enumi}}
\end{enumerate}
\end{multicols}

\enlargethispage{1in}

\begin{enumerate}
\setcounter{enumi}{\value{HW}}


\item  

\begin{enumerate}

\item $p^{-1}(x) = \frac{450-x}{15}$.  The domain of $p^{-1}$ is the range of $p$ which is $[0,450]$

\item  $p^{-1}(105) = 23$. This means that if the price is set to $\$105$ then $23$ dOpis will be sold.

\item $\left(P\circ p^{-1}\right)(x) = -\frac{1}{15} x^2 + \frac{110}{3} x - 5000$, $0 \leq x \leq 450$.  

\smallskip

The graph of $y = \left(P\circ p^{-1}\right)(x)$ is a parabola opening downwards with vertex $\left(275, \frac{125}{3}\right) \approx (275, 41.67)$.  This means that the maximum profit is a whopping $\$41.67$ when the price per dOpi is set to $\$275$.   At this price, we can produce and sell $p^{-1}(275) = 11.\overline{6}$ dOpis.  Since we cannot sell part of a system, we need to adjust the price to sell either $11$ dOpis or $12$ dOpis. We find $p(11) = 285$ and $p(12) = 270$, which means we set the price per dOpi at either $\$285$ or $\$270$, respectively.  The profits at these prices are $\left(P\circ p^{-1}\right)(285) = 35$ and  $\left(P\circ p^{-1}\right)(270) = 40$, so it looks as if the maximum profit is $\$40$ and it is made by producing and selling $12$ dOpis a week at a price of $\$270$ per dOpi.

\end{enumerate}

\addtocounter{enumi}{1}

\item Given that $f(0) = 1$, we have $f^{-1}(1) = 0$.  Similarly $f^{-1}(5) = 1$ and $f^{-1}(-3) = -1$

\addtocounter{enumi}{9}

\item  \begin{enumerate} \addtocounter{enumii}{1} \item If $b =0$, then $m = \pm 1$.  If $b \neq 0$, then $m = -1$ and $b$ can be any real number. \end{enumerate}

\end{enumerate}

\newpage

